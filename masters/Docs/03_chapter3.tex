\chapter{Proponowany algorytm filtrowania}
\thispagestyle{chapterBeginStyle}
\par
Rozdział zawiera opis algorytmu filtrującego dziedziny zmiennych, z których składa sekwencja wejściowa. Dzięki odpowiedniej
modyfikacji metody \textit{RegularExpressionTester} algorytm jest w stanie zebrać informacje dotyczące dopuszczalnych wartości
dla każdej zmiennej. W dalszym procesie dane te są wykorzystane do narzucania możliwych wartości w dziedzinach analizowanych
zmiennych. Po zaakceptowaniu ciągu wejściowego przez automat, zaczynając od stanu końcowego, algorytm iteruje się od końca
sekwencji aż do pierwszej zmiennej kończąc na stanie początkowym. W każdym kroku tej iteracji dla każdej zmiennej zbierane
są wartości, dzięki którym możliwe było przejście do stanu końcowego. Wartości te staną się nową dziedziną dla danej zmiennej,
a dzięki zapisanym informacjom nt. tego, z którego stanu nastąpiło przejście możliwe jest odtworzenie wszystkich możliwych przejść.
\par
Bazując na literaturze oraz analizie nowego algorytmu zebrane zostały twierdzenia dotyczące teoretycznej złożoności czasowej oraz
pamięciowej proponowanej przeze mnie metody. Poddana analizie została również poprawność algorytmu.

\section{Konstrukcja automatu filtrującego dziedziny}
\par
Powyższa metoda może zostać użyta w procesie przetwarzania sekwencji zmiennych o zadanych dziedzinach. Zamiast pojedynczego
symbolu ciągu wejściowego $a_i$ można rozważyć zbiór wartości z dziedziny $D_i$, tj. zaakceptować przejście automatu do kolejnego
stanu, jeśli wymagany do przejścia symbol występuje w dziedzinie zmiennej $X_i$. Głównym celem jest jednak znalezienie możliwości
zredukowania dziedzin zmiennych. W takim przypadku ważnymi danymi, jakie algorytm musi uzyskać są dopuszczone przez automat
wartości dla każdej zmiennej z sekwencji. Aby to uzyskać konieczne jest śledzenie przejść automatu pomiędzy stanami
dla każdej zmiennej wejściowej, więc długość sekwencji staje się kolejnym mnożnikiem w złożoności pamięciowej działania
algorytmu.
\begin{figure}
	\centering
	{\small
		\begin{pseudokod}[H]
		\SetArgSty{normalfont}
		\SetKwProg{Fn}{Function}{}{}
		\SetKwFunction{Enqueue}{Enque}
		\SetKwFunction{Dequeue}{Dequeue}
		\SetKwFunction{Closure}{Closure}
		\SetKwFunction{Transitions}{Transitions}
		\Fn{evaluateAutomatonWithInputVars($Automaton$, $Vars$)}
		{
			$(Q, start, {t}, E)$ $\leftarrow$ $Automaton$ \\
			$currentStates$ $\leftarrow$ \Closure{$E, \{start\}$} \\
			$feasibleValues$ $\leftarrow$ 2D table of size [$n$][$k$] \\
			\ForEach{variable $X_i$ from input vars sequence}
			{
				\ForEach{state $c \in C$}
				{
					\ForEach{state $q$ such that $a \in D_i$ $AND$ $(p, a, q) \in E$}
					{
						$C'$ $\leftarrow$ $C' + \{q\}$ \\
						$feasibleValues[i][c]$ $\leftarrow$ $feasibleValues[i][q] + \{(p, a)\}$ \\
					}
				}
				$currentStates$ $\leftarrow$ \Closure{$E, C'$} \\
			}
			\Return{$C, feasibleValues$}
		}
		\caption{Algorytm bazujący na funkcji \textit{RegularExpressionTester}. Dodanymi przeze mnie krokiem
				 jest zapisywanie historii przejść pomiędzy stanami w czasie procesowania kolejnych zmiennych
				 z sekwencji wejściowej.}
		\label{alg:evaluateAutomatonWithInputVars}
		\end{pseudokod}
	}
\end{figure}

\section{Redukcja dziedzin}
\par
Jeśli zbiór aktualnych stanów będzie pusty zanim wszystkie zmienne zostaną przetworzone to znaczy, że aktualne dziedziny
sekwencji nie są akceptowane przez automat i algorytm propagujący zgłosi naruszenie ograniczenia. Wówczas nastąpi nawrót
(ang. backtracking) w algorytmie przeszukiwania. Taka sama sytuacja wydarzy się w przypadku, gdy przeanalizowany zostanie cały
ciąg wejściowy i zbiór stanów aktywnych nie będzie zawierał stanu końcowego (lub stanu, który jest połączony ścieżką
$\epsilon$-przejść ze stanem końcowym).
\par
Gdy ciąg zmiennych został zaakceptowany przez automat można przystąpić do redukcji dziedzin. Zaczynając od stanu końcowego $s_k$ i
ostatniej zmiennej $D_n$ z sekwencji wejściowej dzięki zapisanym informacjom o tym, z którego stanu i jakim symbolem nastąpiło
przejście można odtworzyć wszystkie możliwe wartości zaakceptowane przez automat dla $D_n$. Oznaczając taki zbiór dopuszczalnych
wartości przez $D'_n$ można z całą pewnością uznać ten zbiór jako nową dziedzinę dla zmiennej $X_n$. Przetwarzając kolejne
stany i kolejne zmienne możliwa jest redukcja dziedzin dla każdej z nich (jeśli zbiór $D'_i$ jest różny od $D_i$).
\par
W czasie przeszukiwania algorytm propagujący ograniczenie zostaje uruchomiony za każdym razem, gdy dziedziny zmiennych
mu podlegających zostają zmodyfikowane na przykład przez działanie propagacji innych ograniczeń bądź przez ustalenie
wartościowania zmiennej (które de facto jest zredukowaniem dziedziny zmiennej do jednej wartości).
\begin{figure}
	\centering
	{\small
		\begin{pseudokod}[H]
		\SetArgSty{normalfont}
		\SetKwProg{Fn}{Function}{}{}
		\SetKwFunction{resetFeasibleValuesVectors}{resetFeasibleValuesVectors}
		\SetKwFunction{evaluateAutomatonWithInputVars}{evaluateAutomatonWithInputVars}
		\SetKwFunction{findFinishStatesAfterSearch}{findFinishStatesAfterSearch}
		\SetKwFunction{reduceVarsDomains}{reduceVarsDomains}
		\SetKwFunction{violateConstraint}{violateConstraint}
		\Fn{RegexFiltering($Automaton$, $Vars$)}
		{
			$statesAfterSearching$, $feasibleValues$ $\leftarrow$ \evaluateAutomatonWithInputVars{$Automaton$, $Vars$} \\
			$finishStates$ $\leftarrow$ \findFinishStatesAfterSearch{$statesAfterSearching$, $Automaton$} \\
			\If{$finishStates$ is empty}
			{
				\violateConstraint{}
			}
			\reduceVarsDomains{$finishStates$, $feasibleValues$, $Vars$}
		}
		\caption{Zarys algorytmu propagacji ograniczenia opisanego wyrażeniem regularnym (źródło własne).}
		\label{alg:RegexFiltering}
		\end{pseudokod}
	}
\end{figure}
\par
Algorytm~\ref{alg:RegexFiltering} korzysta ze zdefiniowanej wcześniej funkcji \textit{evaluateAutomatonWithInputVars} oraz
dwóch dodatkowych \textit{findFinishStatesAfterSearch} i \textit{reduceVarsDomains}. Pierwsza z nich filtruje stany pod kątem tego,
czy w ich domknięciach znajduje się stan końcowy. Jeśli nie ma takiego stanu to oznacza, że ograniczenie nie zostało spełnione.
Druga z wymienionych funkcji jest bardziej warta uwagi, ponieważ to w niej następuje właściwa redukcja dziedzin zmiennych.
\begin{figure}
	\centering
	{\small
		\begin{pseudokod}[H]
		\SetArgSty{normalfont}
		\SetKwProg{Fn}{Function}{}{}
		\SetKwFunction{Insert}{Insert}
		\SetKwFunction{SetDomain}{SetDomain}
		\SetKwFunction{Swap}{Swap}
		\Fn{reduceVarsDomains($finishStates$, $feasibleValues$, $Vars$)}
		{
			\For{$i$ $\leftarrow$ size of variables seq. \KwTo 1}
			{
				$nextStates$ $\leftarrow$ empty set \\
				$goodValues$ $\leftarrow$ empty set \\
				\ForEach{state $\in$ finishStates}
				{
					\ForEach{$(p, a)$ $\in$ $feasibleValues$[$i$][$state$]}
					{
						\Insert{$nextStates$, $p$} \\
						\Insert{$goodValues$, $a$} \\
					}
				}
				\SetDomain{$Vars$[$i$], $goodValues$} \\
				\Swap{$finishStates$, $nextStates$} \\
			}
		}
		\caption{Algorytm redukujący dziedziny zmiennych na podstawie wyników działania automatu $NFA$.}
		\label{alg:reduceVarsDomains}
		\end{pseudokod}
	}
\end{figure}

\section{Poprawność algorytmu}
Algorytm \textit{RegexFiltering} wyklucza te wartości z dziedzin sekwencji zmiennych, które nie będą mogły być wykorzystane do 
poprawnego wartościowania zapewniającego akceptację ciągu przez automat. 
\begin{theorem}
Proponowany algorytm filtrujący \textit{RegexFiltering} zawsze się zatrzymuje.
\end{theorem}
\begin{proof}
Procedura \textit{RegexFiltering} (patrz kod \ref{alg:RegexFiltering}) wywołuje sekwencyjnie trzy funkcje, którym należy się przyjrzeć. Funkcja
\textit{evaluateAutomatonWithInputVars} zawiera trzy zagnieżdżone pętle \textit{for} o ściśle określonych warunkach zakończenia,
które w czasie działania tej funkcji są niezmienne. Kolejnym krokiem jest funkcja \textit{findFinishStatesAfterSearch}, która
jedynie wyszukuje stany końcowe w domknięciu wszystkich stanów, w jakich zatrzymał się automat. Ostatnia \textit{reduceVarsDomains}
podobnie jak pierwsza z omawianych zawiera zagnieżdżone pętle \textit{for}, których warunki zakończenia są niezależne od jej
działania i są zdefiniowane w poprawny sposób.
\end{proof}
\begin{theorem}
Proponowany algorytm filtrujący \textit{RegexFiltering} nigdy nie usunie dopuszczalnego rozwiązania.
\end{theorem}
\begin{proof}
Wynika to z samej zasady działania algorytmu. Po przeanalizowaniu całego wejścia przez automat następuje przetwarzanie
wypełnionej struktury \textit{feasibleValues} zaczynając od stanu końcowego aż do stanu początkowego.
W ten sposób algorytm będzie poruszał się jedynie po ścieżkach zaakceptowanych przez automat, nie omijając żadnych dopuszczalnych
wartości.
\end{proof}
\par
Z tego samego twierdzenia wynika, że \textit{RegexFiltering} zawsze poprawnie redukuje dziedziny - każde możliwe dopuszczalne
rozwiązanie w czasie każdej iteracji znajduje się w przefiltrowanym zbiorze dziedzin $D'_1 \times ... \times D'_n$.

\section{Złożoność obliczeniowa algorytmu}
\begin{theorem}
Złożoność czasowa algorytmu \textit{evaluateAutomatonWithInputVars} wynosi $O(n*k^2)$, a złożoność pamięciowa $O(n*k)$.
\label{theory:evaluateAutomatonWithInputVars}
\end{theorem}
\begin{proof}
Pod względem złożoności czasowej, różnicą pomiędzy \textit{evaluateAutomatonWithInputVars} a metodą
\textit{RegularExpressionTester} jest zapisywanie historii przejść pomiędzy stanami do tablicy \textit{feasibleValues}.
Ponieważ zapis ten odbywa się w czasie $O(1)$ jest on zaniedbywalny w kontekście asymptotycznej złożoności.
Jednakże, przez wprowadzenie tzw. $\epsilon$-przejść (patrz rozdział 1.) w automacie, w pesymistycznym przypadku
w czasie procesowania każdego stanu $S_i$ może okazać się, że konieczne jest przetworzenie dodatkowo $k-i$
stanów osiągalnych przez $\epsilon$-ścieżki. To oznacza, że dla $k$ stanów algorytm może przeanalizować
dla danej zmiennej $X_j$:
\begin{equation}
	\begin{aligned}
		\sum_{i=1}^{k}{(k-i)} = \frac{1}{2}(k-1)k \in O(k^2)
	\end{aligned}
\end{equation}
Z tego wynika, że przetworzenie całego ciągu wejściowego kosztuje:
\begin{equation}
	\begin{aligned}
		\sum_{j=1}^{n}{\sum_{i=1}^{k}{(k-i)}} \in O(n*k^2)
	\end{aligned}
\end{equation}
\par
Zwiększeniu względem \textit{RegularExpressionTester} podlega też złożoność pamięciowa, ponieważ algorytm ten wymaga tablicy wielkości $n*k$ wektorów
par $(p, a)$, gdzie $p$ oznacza stan z którego nastąpiło przejście, a $a$ symbol przejścia. Każdy wektor par
może w danym przebiegu algorytmu przechować maksymalnie $l = k*\alpha$, gdzie $\alpha$ jest stałą oznaczającą maksymalną
liczbę przejść, jaka może wyjść z jednego stanu. Liczba par jaka może zostać dodana do $feasibleValues[i]$ w czasie
przetwarzania zmiennej $X_i$ to również $k*\alpha$, a więc zamortyzowany koszt pamięciowy jednego wektora par dla $X_i$
wynosi $l' = \frac{k*\alpha}{k} = \alpha$, a więc $O(1)$. To oznacza, że wielkości wektorów par nie wpływają na asymptotyczną
złożoność pamięciową.
\end{proof}
\begin{theorem}
Złożoność czasowa algorytmu \textit{reduceVarsDomains} wynosi $O(n*k)$.
\label{theory:reduceVarsDomains}
\end{theorem}
\begin{proof}
Algorytm~\ref{alg:reduceVarsDomains} iteracyjnie przechodzi po wszystkich zmiennych ciągu wejściowego oraz po zbiorze
aktualnych stanów dodając do zbioru następnych stanów kolejne wpisy, których jest maksymalnie $O(k)$. Zakładając, że
implementacja zbioru pozwala na dodanie wpisu w zaniedbywalnym czasie to po złożeniu tych funkcji wynikiem jest $O(n*k)$.
\end{proof}
\par
Korzystając z powyższych twierdzeń można wyprowadzić wniosek:
\begin{theorem}
Złożoność czasowa algorytmu \textit{RegexFiltering} wynosi $O(n*k^2)$.
\label{theorem:RegexFiltering}
\end{theorem}
\begin{proof}
Złożoność algorytmu \textit{RegexFiltering} można uzyskać przez zsumowanie złożoności poszczególnych metod składowych, tj.
(patrz algorytmy \ref{theory:evaluateAutomatonWithInputVars} i \ref{theory:reduceVarsDomains}) oraz \textit{findFinishStatesAfterSearch}.
W wyniku daje to $O(n*k^2) + O(n*k) + O(k) = O(n*k^2)$.
\end{proof}
