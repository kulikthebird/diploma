\chapter{Podstawowe pojęcia}
\thispagestyle{chapterBeginStyle}
\label{rozdzial1}
Rozdział poświęcony jest między innymi krótkiemu wprowadzeniu w tematykę programowania z ograniczeniami, działania
algorytmów filtrujących oraz ich miejsca w modelowaniu. Omówiono również podstawowe zagadnienia z dziedziny języków
formalnych jakimi są języki regularne oraz praktyczne sposoby ich opisu. Połączenie powyższych zagadnień pozwala
na opis globalnych ograniczeń przy pomocy wyrażeń regularnych na sekwencjach zmiennych o określonych dziedzinach.


\section{Programowanie z ograniczeniami}
\par
Nawiązując do książki "Handbook of Constraints Programming"~\cite{HandbookOfCP}, programowanie z ograniczeniami to
paradygmat służący rozwiązywaniu kombinatorycznych problemów przeszukiwania, pojawiających się w wielu dziedzinach
techniki m. in. w sztucznej inteligencji, szeroko pojętej informatyce oraz badaniach operacyjnych.
\par
Formalnie problem spełnienia ograniczeń $P$ to taka trójka $(X, D, C)$, gdzie 
\begin{itemize}
    \item $X$ - n-elementowa krotka $(X_1, X_2, ... , X_n)$ zmiennych,
    \item $D$ - n-elementowa krotka $(D_1, D_2, ... , D_n)$ zbiorów stanowiących dziedziny odpowiadających zmiennych z
                krotki X, tj. $X_i \in D_i$,
    \item $C$ - t-elementowa krotka $(C_1, C_2, ... , C_t)$ ograniczeń narzuconych na zmienne z krotki $X$.
\end{itemize}
Ograniczenia są to relacje pomiędzy zmiennymi. Innymi słowy ograniczenie $C_i$ jest to para $(R_{Si}, S_i)$, gdzie
\begin{itemize}
    \item $S_i$ jest krotką złożoną ze zmiennych krotki $X$,
    \item $R_{Si}$ jest podzbiorem iloczynu kartezjańskiego dziedzin zmiennych
        z krotki $S_i$, tj. $R_{Si} \subseteq (D_1^{Si} \times D_2^{Si} \times ... \times D_k^{Si})$.
\end{itemize}
Rozwiązaniem problemu $P$ jest n-elementowa krotka $A = (a_1, a_2, ... , a_n)$, w której każde $a_i \in D_i$ oraz
wszystkie narzucone ograniczenia zostały spełnione - $(\forall_{(R, S) \in C}) \pi_S(A) \in R$, gdzie
$\pi_S(A)$ jest to rzut krotki $A$ zawierający zmienne ze zbioru $S$. Zbiorem $sol(P)$ nazywa się zbiór rozwiązań
dopuszczalnych dla problemu $P$ zawierającym wszystkie możliwe rozwiązania spełniające ograniczenia.
\par
W dalszej części rozważany będzie również zbiór $\Omega = D_1 \times D_2 \times ... \times D_n$.
Jest to zbiór wszystkich krotek, w których wartość każdej zmiennej przynależy do odpowiedniej dziedziny.
Innymi słowy jest to zbiór dopuszczalnych wyników ze względu na dziedziny zmiennych.
\par
\begin{example}
\textbf{Problem spełnienia ograniczeń:}
\begin{equation}
    \left\{
        \begin{aligned}
            X &= (A, B, C), \\
            D &= (\{2,3\}, \{3,4\}, \{8,16,32\})
        \end{aligned}
    \right.
\end{equation}
Powyżej zdefiniowane zostały zmienne oraz ich dziedziny. Następnym krokiem jest określenie ograniczeń na tych zmiennych:
\begin{equation}
    \left\{
        \begin{aligned}
            3A &> C \\
            3B &> C \\
            A &\neq B,
        \end{aligned}
    \right.
\end{equation}
Jedynym dopuszczalnym rozwiązaniem dla powyższego problemu jest trójka: $(A=3, B=4, C=8)$.
\end{example}

\section{Poszukiwanie rozwiązania}
\par
Podstawą działania algorytmów rozwiązujących problemy programowania z ograniczeniami są:
\begin{itemize}
    \item Przeszukiwanie
    \item Wnioskowanie
\end{itemize}
Pierwsze z nich polega kolejnym generowaniu potencjalnych wektorów wynikowych 
$A' \in \Omega$. Jeśli krotka $A'$ spełnia wszystkie narzucone ograniczenia z $C$
to staje się jednym z potencjalnie wielu rozwiązań danego problemu. Stosuje się różne techniki
oraz heurystyki poszukiwania rozwiązania.
\par
Łatwo zauważyć, że przetestowanie wszystkich
możliwych krotek wymaga $|\Omega| = |D_1| * |D_2| * ... * |D_n|$ prób. Celem zredukowania mocy zbioru $\Omega$
stosuje się metodę wnioskowania, tj. usuwania z dziedzin zmiennych tych wartości, które z pewnością nie mogą zostać
wybrane ze względu na narzucone ograniczenia. Taki proces nazywa się propagacją ograniczeń. Również w tej kwestii
rozważyć można wiele różnych technik. W ogólności im bardziej algorytm jest w stanie zawęzić dziedziny tym
mniej pracy będzie miał algorytm przeszukiwania, co najczęściej przyspiesza znalezienie wyniku.
\par
Połączenie wyżej wspomnianych technik skutkuje szybszym znalezieniem rezultatów. Iteracyjne stosowanie przeszukiwania
z nawrotami oraz propagacji ograniczeń zawęża drzewo przeszukiwania, co jest ważnym atutem w przypadku problemów
NP-trudnych, do których najczęściej stosuje się technikę programowania z ograniczeniami.

\section{Propagacja ograniczeń}
W przypadku algorytmów przeszukiwania można uciekać się do stosowania różnych heurystycznych metod. Jednak
to zawężanie zbioru dopuszczalnych rozwiązań stanowi główną zaletę algorytmu. Podstawową metodą redukcji dziedzin
jest propagacja przy użyciu tzw. Network Consistency. Ta metoda pozwala zachować lokalną spójność pomiędzy
dziedzinami zmiennych.
\par
Niech zmienne z problemu $P$ będą reprezentowane przez wierzchołki w grafie. Pierwszą z technik jest wprowadzenie
spójności wierzchołkowej (ang. Node consistency). Polega to redukcji dziedzin zmiennych tak, by spełniały ograniczenia
jednoargumentowe dla nich samych, tj. $D'_i = D_i \cap R_j$, dla każdego $R_j$ będącego ograniczeniem dla zmiennej $X_i$. 
Krawędzie pomiędzy wierzchołkami reprezentują ograniczenia binarne pomiędzy parami zmiennych. Dla ograniczenia pomiędzy
zmiennymi $X_i$ oraz $X_j$ - $R_{ij}$ należy rozpatrzeć spójność krawędziową - ang. Arc consistency.
W tym celu wykluczyć można te wartości dziedzin zmiennych, które nie należą do relacji, tj. wyznaczyć
$D'_i = D_i \cap \pi_i(R_{ij})$ oraz analogicznie $D'_j = D_j \cap \pi_j(R_{ij})$. Po zredukowaniu dziedziny
danej zmiennej może okazać się, że w grafie powstały kolejne niespójności, więc proces ten należy powtarzać iteracyjnie
do momentu osiągnięcia pełnej spójności grafu.
\par
Kolejnym możliwym etapem propagacji ograniczeń jest spójność ścieżek - ang. Path consistency. Analogicznie jak w przypadku
spójności krawędziowej można przeanalizować spójność relacji dwóch zmiennych $R_{ij}$ względem dziedziny trzeciej zmiennej
$X_m$. Polega to na wykluczaniu z $R_{ij}$ takich par dopuszczalnych wartości, które nie pozwalają zmiennej $X_m$
przypisać żadnej wartości.

\section{Globalna spójność}
\par
Osiągnięcie ogólnej krawędziowej spójności jest problemem NP-trudnym~\cite{HandbookOfCP}. Jednak w wielu
przypadkach możliwe jest wprowadzenie wyspecjalizowanych algorytmów o niższej złożoności czasowej i/lub
pamięciowej. Za sztandarowy przykład można uznać ograniczenie \textit{all\_different}, narzucające na zmienne warunek, aby
wartości tych zmiennych były parami różne - $\forall_{x'_1, x'_2 \in all\_diff(x_1,...,x_k)} x'_1 \neq x'_2$.
\begin{example}
\textbf{Problem all\_different:}
\begin{equation}
    \begin{cases}
        \begin{aligned}
            X &= (A, B, C), \\
            D &= (\{1, 2, 3\}, \{1, 2, 3\}, \{1, 2, 3\}), \\
            &\text{pod warunkiem, że:} \\
            &\text{all\_different($A$, $B$, $C$).}
        \end{aligned}
    \end{cases}
\end{equation}
Zbiorem rozwiązań są wszystkie permutacje trójki: $(1, 2, 3)$.
\par
\end{example}
W celu zrozumienia sensu wprowadzania dodatkowych algorytmów dla \textit{all\_different} należy rozpatrzeć następny przykład.
\begin{example}
\textbf{Problem zmiennych o parami różnych wartościach:}
\begin{equation}
    \begin{cases}
        \begin{aligned}
            X &= (A, B, C, D), \\
            D &= (\{1, 2, 3\}, \{1, 2, 3\}, \{1, 2, 3\}, \{1, 2, 3\}), \\
            &\text{pod warunkiem, że:} \\
            A &\neq B, A \neq C, A \neq D, \\
            B &\neq C, B \neq D, C \neq D.
        \end{aligned}
    \end{cases}
\end{equation}
Zbiór rozwiązań jest pusty, ponieważ nie jest możliwe takie wartościowanie, które dla każdej pary zmiennych przyporządkuje różne
wartości. Zauważyć można, że spójność wierzchołkowa wraz z krawędziową są zachowane, więc nie nastąpi redukcja dziedzin.
Analiza spójności ścieżek również nie wykaże problemów. Algorytm przeszukiwania przetestuje więc cały zbiór $\Omega$ zanim
stwierdzi, że problem nie ma rozwiązania. W tym przypadku dopiero sprawdzenie 4-spójności wszystkich kombinacji trzech zmiennych
$<X_i, X_j, X_m>$ względem czwartej zmiennej $X_k$ pozwoliłoby zauważyć sprzeczność, jednak w ogólności badanie n-spójności jest
problemem trudnym pod względem złożoności. W przypadku \textit{all\_different} istnieje jednak algorytm o wielomianowym czasie 
wykonywania, który skutecznie propaguje ograniczenia oraz jest w stanie stwierdzić sprzeczność na etapie samej propagacji~
\cite{Filtering}. Działa on w oparciu o rozwiązanie problemu na grafie dwudzielnym.
\end{example}

\section{Języki regularne}
Języki regularne jest to rodzina języków akceptowanych przez automaty skończone (o skończonej liczbie stanów).
Gramatykę można opisać w następujący sposób:
\begin{equation}
    \begin{aligned}
        A &\rightarrow a \text{ - symbol \textit{a} jest dowolnym terminalem} \\
        A &\rightarrow aB \text{ - symbol \textit{a} jest dowolnym terminalem, a \textit{B} nieterminalem} \\
        A &\rightarrow \epsilon \text{ - gdzie $\epsilon$ jest pustym łańcuchem znaków}
    \end{aligned}
\end{equation}
\par
Innym sposobem przedstawiania gramatyki regularnej są tzw. wyrażenia regularne (ang. regular expressions).
Przy pomocy konkatenacji, alternatywy oraz domknięcia Kleene'ego możliwe jest opisanie gramatyki języka regularnego:
\begin{itemize}
    \item $ab$ - konkatenacja oznacza, że po znaku $a$ musi pojawić się znak $b$
    \item $a|b$ - alternatywa wykluczająca, tj. w danym miejscu może pojawić się albo $a$ albo $b$
    \item $a^*$ - domknięcie Kleene'ego oznacza 0 lub więcej znaków $a$
\end{itemize}
Wyrażenia regularne wzbogacone są również o inne operatory, które w znacznym stopniu ułatwiają zapis wyrażenia, np.
\begin{itemize}
    \item $a^? \equiv (a|\epsilon)$
    \item $a^+ \equiv (aa^*)$
    \item $[a-z0-9] \equiv (a|b|c|...|z|0|1|...|9)$
    \item $[\string^a]$ - dowolny znak poza symbolem $a$
\end{itemize}
\par
Automaty skończone można podzielić na dwie klasy - automat deterministyczny ($DFA$ - ang. deterministic finite automaton)
oraz automat niedeterministyczny ($NFA$ - ang. non-deterministic finite automaton). Moc obu klas jest równa, co oznacza,
że każdą gramatykę regularną można przedstawić zarówno w postaci automatu $DFA$ jak i $NFA$~\cite{FormalLanguages1}.
Deterministyczny automat skończony jest to piątka $(Q, \Sigma, \delta, s, F)$:
\begin{itemize}
    \item $Q$ jest skończonym zbiorem stanów
    \item $\Sigma$ to alfabet wejściowy
    \item $\delta: Q \cap \Sigma \leftarrow Q$ jest funkcją przejścia pomiędzy stanami
    \item $s \in Q$ jest stanem startowym
    \item $F \subseteq Q$ jest zbiorem stanów akceptujących
\end{itemize}
Automaty skończone można przedstawić w postaci grafu, w którym wierzchołki oznaczają stany, a krawędzie
przejścia pomiędzy stanami (pod warunkiem wystąpienia odpowiedniego symbolu/symboli).
\begin{figure}
    \centering
    \resizebox{0.40\textwidth}{!}{
        \begin{tikzpicture}[>=stealth',shorten >=1pt,auto,node distance=1.0cm]
            \node[state,initial] (q_0)   {$q_0$}; 
            \node[state] (q_1) [above right=of q_0] {$q_1$}; 
            \node[state] (q_2) [below right=of q_0] {$q_2$}; 
            \node[state,accepting](q_3) [below right=of q_1] {$q_3$};
            \path[->] 
                (q_0) edge  node  {0} (q_1)
                    edge  node  [swap] {1} (q_2)
                (q_1) edge  node  {1} (q_3)
                    edge  [loop above] node {0} ()
                (q_2) edge  node [swap] {0} (q_3) 
                    edge  [loop below] node {1} ();
        \end{tikzpicture}
    }
    \caption{Przykład automatu skończonego, który jest równoważny wyrażeniu regularnemu: $(00^{*}1)|(11^{*}0)$}
\end{figure}
\par
Z formalnego punktu widzenia niedeterministyczny automat skończony $NFA$ z definicji różni się od $DFA$ funkcją
przejścia. W przypadku niedeterministycznej wersji występuje $\delta: Q \cap \Sigma \rightarrow 2^Q$, gdzie
$2^Q$ oznacza zbiór potęgowy stanów automatu. Oznacza to, że z danego stanu i symbolu wejściowego można
niedeterministycznie przejść do każdego ze stanów określonego podzbioru $Q$. Istnieje co najmniej kilka
metod radzenia sobie ze złożonością obliczeniową automatów $NFA$, co zostanie poruszone w następnym rozdziale.

\par
Symbol $\epsilon$ oznacza pusty łańcuch znaków. Oznacza to możliwość bezwarunkowego skoku do kolejnego stanu bez
pobrania symbolu z taśmy wejściowej. Przejście pomiędzy dwoma stanami $S_i \rightarrow S_j$ pod warunkiem wystąpienia
$\epsilon$ będzie w dalszej części pracy określone przez $\epsilon$-przejście, a ciąg możliwych przejść z danego stanu
$S_i$ do $S_j$ oznaczany
$S_i \rightarrow S_{k1} \rightarrow S_{k2} \rightarrow ... \rightarrow S_j \equiv S_i \rightarrow^* S_j$ będzie nazywane
$\epsilon$-ścieżką.

\section{Wyrażenia regularne jako opis globalnych ograniczeń}
\par
W literaturze można spotkać różne nazwy określające algorytmy propagacji ograniczeń jak np. algorytmy filtrujące,
globalne ograniczenia itp. Powstały również zbiory zawierające opisy działania oraz sposoby implementacji poszczególnych
globalnych ograniczeń. Jednym ze sposobów na opisanie rozwiązań dopuszczalnych dla zadanej sekwencji zmiennych jest
użycie automatu skończonego~\cite{Automata}.
Sekwencje akceptowane przez automat byłyby dopuszczalnymi rozwiązaniami dla danego globalnego ograniczenia, a sekwencja
niezaakceptowana oznaczałaby niespełnienie ograniczenia i w konsekwencji natychmiastowy nawrót algorytmu przeszukującego.
Jednocześnie na podstawie samego automatu można ograniczać dziedziny zmiennych, co pozwala na skuteczniejszą propagację
ograniczenia.
\par
Każdy automat skończony można przedstawić w postaci wyrażenia regularnego oraz każde wyrażenie regularne możne zostać
przedstawione w formie automatu skończonego~\cite{FormalLanguages2}. Zatem dla niektórych globalnych ograniczeń wygodnie
byłoby opisać ich działanie w formie wyrażenia regularnego.
\par
Jako dobry przykład może posłużyć problem rozmieszczania rozłącznych odcinków w wektorze zmiennych binarnych.
Załóżmy, że $X = (X_1, X_2, ... , X_n)$ oraz, że każda zmienna $X_i \in \{ 0, 1 \} $. Niech w sekwencji $X$ rozmieszczone
będzie $k$ odcinków o zadanych długościach $I = (I_1, I_2, ..., I_k)$ wartościowanych jedynkami, a pomiędzy odcinkami 
zmienne będą wartościowane zerem. Kolejność wystąpienia odcinków jest taka jak w wektorze $I$. 
\par
\begin{example}
\textbf{Problem rozmieszczania odcinków:}
\begin{equation}
    \begin{cases}
        \begin{aligned}
            X &= (X_1, X_2, X_3, X_4, X_5, X_6, X_7), \\
            D &= (\{0, 1\}, ... , \{0, 1\}), \\
            &\text{Niech w wektorze $X$ rozmieszczone będą odcinki $(2, 3)$.}
        \end{aligned}
    \end{cases}
\end{equation}
\par
Dopuszczalne rozwiązania powyższego problemu to:
\begin{equation}
    \begin{aligned}
        &(1, 1, 0, 1, 1, 1, 0) \\
        &(1, 1, 0, 0, 1, 1, 1) \\
        &(0, 1, 1, 0, 1, 1, 1) \\
    \end{aligned}
\end{equation}
\par
\end{example}
\par
Nawiązując do automatów skończonych, dla powyższego przykładu można skonstruować wyrażenie regularne na sekwencji $X$
opisujące ograniczenia na rozmieszczanie zadanych odcinków: $ 0^{*} 1^2 0^{+} 1^2 0^{*} $. W ogólności sekwencję
$I = (I_1, I_2, ..., I_k)$ można skonstruować na podstawie wzoru
$ 0^{*} (1^{I_1} 0^{+}) (1^{I_2} 0^{+}) ... 1^{I_k} 0^{*} $. To oznacza, że dla każdego przykładu powyższe ograniczenie
można zapisać w postaci wyrażenia regularnego, a co za tym idzie skonstruować automat $NFA$. Dalsza część pracy skupiona
jest na przetwarzaniu ciągu znaków wyrażenia regularnego na automat $NFA$ oraz zoptymalizowanej metodzie przetwarzania
sekwencji zmiennych (ich dziedzin) używając tego automatu. Przedstawiona została również metoda propagacji
ograniczeń na podstawie stworzonego automatu.
