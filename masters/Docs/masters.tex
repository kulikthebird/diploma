% ***************************************************************************************************
%
%	Praca inżynierska:
%	Autor:	Tomasz Kulik
%
% ***************************************************************************************************

% Styl dwustronny z domyślną wielkością czcionki 10pt oraz oddzieloną stroną tytułową (titlepage).
% Domyślnie rodziały rozpoczynają się na stronie prawej (openright).
\documentclass{book}
\usepackage[polish]{babel}

% Pakiet babel dla polskiego języka powoduje konflikt z pakietem amssymb.
% Polecenie '\lll' definiują oba pakiety - porządana jest druga definicja.
\let\lll\undefined

\selectlanguage{polish}
\usepackage[utf8]{inputenc}
\usepackage{lmodern}
\usepackage{indentfirst}
\usepackage[plmath]{polski}
\usepackage{icomma}
\frenchspacing
\usepackage{amsmath}
\usepackage{amssymb}
\usepackage{mathrsfs}
\usepackage{bbold}      % Dodatkowe wsparcie dla środowiska mathbb, które nie wspiera domyślnie cyfr (\mathbb{})
\usepackage{mathtools}
\usepackage[table]{xcolor}% http://ctan.org/pkg/
\usepackage{tabu}       % Umożliwia łatwą zmianę koloru linii w tabeli
\usepackage{xcolor}     % Umożliwia rozszerzoną kontrolę nad kolorami.
\definecolor{lgray}{HTML}{9F9F9F}
\definecolor{dgray}{HTML}{5F5F5F}
\usepackage[ruled,vlined,linesnumbered,longend,algochapter]{algorithm2e}        % Udostępnia środowisko do konstruowania pseudokodów
% ruled	- poziome kreski na początku i końcu algorytmu, podpis na górze oddzielony również kreską poziomą
% vlined - pionowe kreski łączące początek polecenia z jego końcem
% linesnumbered	- numerowanie kolejnych wierszy algorytmu
% longend - długie końcówki np. ifend, forend itd.
% algochapter - numeracja z rozdziałami

% Zamiana nazwy środowiska z domyślnej "Algorithm X" na "Pseudokod X"
\newenvironment{pseudokod}[1][htb]{
        \renewcommand{\algorithmcfname}{Pseudokod}
        \begin{algorithm}[#1]%
        }{
\end{algorithm}
}

% Zmiana rozmiaru komentarzy
\newcommand\algcomment[1]{
        \footnotesize{#1}
}

% Ustawienie zadanego stylu dla komentarzy
\SetCommentSty{algcomment}
\usepackage{textcomp}%
\newcommand{\textapprox}{\raisebox{0.5ex}{\texttildelow}}
\usepackage{listings}
\renewcommand{\lstlistingname}{Kod źródłowy}

% Definicje pecjalnych znaków, które nie są obsługiwane w środowisku listing
\lstset{literate=
        {ż}{{\.{z}}}1	{ź}{{\'{z}}}1
        {ć}{{\'{c}}}1	{ń}{{\'{n}}}1
        {ą}{{\c a}}1	{ś}{{\'{s}}}1
        {ł}{{\l}}1		{ę}{{\c{e}}}1
        {ó}{{\'{o}}}1	{á}{{\'a}}1
        {é}{{\'e}}1		{í}{{\'i}}1
        {ó}{{\'o}}1		{ú}{{\'u}}1
        {ù}{{\`u}}1		{Á}{{\'A}}1
        {É}{{\'E}}1		{Í}{{\'I}}1
        {Ó}{{\'O}}1		{Ú}{{\'U}}1
        {à}{{\`a}}1		{è}{{\'e}}1
        {ì}{{\`i}}1		{ò}{{\`o}}1
        {ò}{{\`o}}1		{À}{{\`A}}1
        {È}{{\'E}}1		{Ì}{{\`I}}1
        {Ò}{{\`O}}1		{Ò}{{\`O}}1
        {ä}{{\"a}}1		{ë}{{\"e}}1
        {ï}{{\"i}}1		{ö}{{\"o}}1
        {ü}{{\"u}}1		{Ä}{{\"A}}1
        {Ë}{{\"E}}1		{Ï}{{\"I}}1
        {Ö}{{\"O}}1		{Ü}{{\"U}}1
        {â}{{\^a}}1		{ê}{{\^e}}1
        {î}{{\^i}}1		{ô}{{\^o}}1
        {û}{{\^u}}1		{Â}{{\^A}}1
        {Ê}{{\^E}}1		{Î}{{\^I}}1
        {Ô}{{\^O}}1		{Û}{{\^U}}1
        {œ}{{\oe}}1		{Œ}{{\OE}}1
        {æ}{{\ae}}1		{Æ}{{\AE}}1
        {ß}{{\ss}}1		{ç}{{\c c}}1
        {Ç}{{\c C}}1	{ø}{{\o}}1
        {å}{{\r a}}1	{Å}{{\r A}}1
        {€}{{\EUR}}1	{£}{{\pounds}}1
}

% Ustawienia rozmiarów stron i ich marginesów
\usepackage[headheight=18pt, top=25mm, bottom=25mm, left=25mm, right=25mm]{geometry}
% headheight	-	wysokość tytułów
% top	        -	margines górny
% bottom	-	margines dolny
% left		-	margines lewy
% right		-	margines prawy

% Usunięcie górnego marginesu dla środowisk
\makeatletter
\setlength\@fptop{0\p@}
\makeatother

\usepackage{titling}                    % zapewnia pełną kontrolę nad układem strony tytułowej.
\usepackage{tocloft}                    % Umożliwia modyfikowanie stylu spisu treści
\tocloftpagestyle{tableOfContentStyle}
\usepackage{fancyhdr}                   % Definiowanie własnych stylów nagłówków i/lub stopek

% Domyślny styl dla pracy
\fancypagestyle{custom}{
        \fancyhf{}								% wyczyść stopki i nagłówki
        \fancyhead[RO]{								% Prawy, nieparzysty nagłówek
                \hrulefill \hspace{16pt} \large Rozdział \thechapter
                \put(-472.1, 12.1){%
                        \makebox(0,0)[l]{%
                                \includegraphics[width=0.05\textwidth]{graphics/pwr-logo}
                        }
                }
                \put(-443,5.5){%
                        \makebox(0,0)[l]{%
                                \small Politechnika Wrocławska
                        }
                }
        }
        \fancyhead[LE]{								% Lewy, parzysty nagłówek
                \large Rozdział \thechapter \hspace{16pt} \hrulefill
                \put(-22, 12.1){%
                        \makebox(0,0)[l]{%
                                \includegraphics[width=0.05\textwidth]{graphics/wppt-logo}
                        }
                }
                \put(-210,5.5){%
                        \makebox(0,0)[l]{%
                                \small Wydział Podstawowych Problemów Techniki
                        }
                }
        }
        \fancyfoot[LE,RO]{							% Stopki
                \thepage
        }
        \renewcommand{\headrulewidth}{0pt}			% Grubość linii w nagłówku
        \renewcommand{\footrulewidth}{0.2pt}		% Grubość linii w stopce
}

% Domyślny styl dla bibliografii
\fancypagestyle{bibliographyStyle}{
        \fancyhf{}									% wyczyść stopki i nagłówki
        \fancyhead[RO]{								% Prawy, nieparzysty nagłówek
                \hrulefill \hspace{16pt} \large Dodatek \thechapter
                \put(-472.1, 12.1){%
                        \makebox(0,0)[l]{%
                                \includegraphics[width=0.05\textwidth]{graphics/pwr-logo}
                        }
                }
                \put(-443,5.5){%
                        \makebox(0,0)[l]{%
                                \small Politechnika Wrocławska
                        }
                }
        }
        \fancyhead[LE]{								% Lewy, parzysty nagłówek
                \large Bibliografia \hspace{16pt} \hrulefill
                \put(-22, 12.1){%
                        \makebox(0,0)[l]{%
                                \includegraphics[width=0.05\textwidth]{graphics/wppt-logo}
                        }
                }
                \put(-210,5.5){%
                        \makebox(0,0)[l]{%
                                \small Wydział Podstawowych Problemów Techniki
                        }
                }
        }
        \fancyfoot[LE,RO]{							% Stopki
                \thepage
        }
        \renewcommand{\headrulewidth}{0pt}			% Grubość linii w nagłówku
        \renewcommand{\footrulewidth}{0.2pt}		% Grubość linii w stopce
}

% Domyślny styl dla dodatków
\fancypagestyle{appendixStyle}{
        \fancyhf{}									% wyczyść stopki i nagłówki
        \fancyhead[RO]{								% Prawy, nieparzysty nagłówek
                \hrulefill \hspace{16pt} \large Dodatek \thechapter
                \put(-472.1, 12.1){%
                        \makebox(0,0)[l]{%
                                \includegraphics[width=0.05\textwidth]{graphics/pwr-logo}
                        }
                }
                \put(-443,5.5){%
                        \makebox(0,0)[l]{%
                                \small Politechnika Wrocławska
                        }
                }
        }
        \fancyhead[LE]{								% Lewy, parzysty nagłówek
                \large Dodatek \thechapter \hspace{16pt} \hrulefill
                \put(-22, 12.1){%
                        \makebox(0,0)[l]{%
                                \includegraphics[width=0.05\textwidth]{graphics/wppt-logo}
                        }
                }
                \put(-210,5.5){%
                        \makebox(0,0)[l]{%
                                \small Wydział Podstawowych Problemów Techniki
                        }
                }
        }
        \fancyfoot[LE,RO]{							% Stopki
                \thepage
        }
        \renewcommand{\headrulewidth}{0pt}			% Grubość linii w nagłówku
        \renewcommand{\footrulewidth}{0.2pt}		% Grubość linii w stopce
}

% Osobny styl dla stron zaczynających rozdział/spis treści itd. (domyślnie formatowane jako "plain")
\fancypagestyle{chapterBeginStyle}{
        \fancyhf{}%
        \fancyfoot[LE,RO]{
                \thepage
        }
        \renewcommand{\headrulewidth}{0pt}
        \renewcommand{\footrulewidth}{0.2pt}
}

% Styl dla pozostałych stron spisu treści
\fancypagestyle{tableOfContentStyle}{
        \fancyhf{}%
        \fancyfoot[LE,RO]{
                \thepage
        }
        \renewcommand{\headrulewidth}{0pt}
        \renewcommand{\footrulewidth}{0.2pt}
}

% Formatowanie tytułów rozdziałów i/lub sekcji
\usepackage{titlesec}

% Formatowanie tytułów rozdziałów
\titleformat{\chapter}[hang]					% kształt
{
        \vspace{-10ex}
        \Huge
        \bfseries
}												% formatowanie tekstu modyfikowanego elementu
{}												% etykieta występująca przed tekstem modyfikowanego elementu, niewidoczna w spisie treści
{
        10pt
}												% odstęp formatowanego tytułu od lewego marginesu/etykiety
{
        \Huge
        \bfseries
}               % formatowanie elementów przed modyfikowanym tytułem
[
\vspace{2ex}
%\rule{\textwidth}{0.4pt}
%\vspace{-4ex}
]												% dodatkowe formatowanie stosowane poniżej modyfikowanego tytułu


% Formatowanie tytułów sekcji
\titleformat{\section}[hang]					% kształt
{
        \vspace{2ex}
%	\titlerule\vspace{1ex}
        \Large\bfseries
}												% formatowanie tekstu modyfikowanego elementu
{
        \thesection									% etykieta występująca przed tekstem modyfikowanego elementu, niewidoczna w spisie treści
}
{
        0pt
}												% odstęp formatowanego tytułu od lewego marginesu/etykiety
{
        \Large
        \bfseries
}% formatowanie elementów przed modyfikowanym tytułem


\usepackage{hyperref}	% Aktywuje linki
\hypersetup{
        colorlinks	=	true,					% Koloruje tekst zamiast tworzyć ramki.
        linkcolor		=	blue,					% Kolory: referencji,
        citecolor		=	blue,					% cytowań,
        urlcolor		=	blue					% hiperlinków.
}
\urlstyle{same} % Do stworzenia hiperłączy zostanie użyta ta sama (same) czcionka co dla reszty dokumentu
\usepackage{enumitem}   % Umożliwia zdefiniowanie własnego stylu wyliczeniowego
\newlist{myitemize}{itemize}{3}         % Nowa lista numerowana z trzema poziomami

% Definicja wyglądu znacznika pierwszego poziomu
\setlist[myitemize,1]{
        label		=	\textbullet,
        leftmargin	=	4mm}

% Definicja wyglądu znacznika drugiego poziomu
\setlist[myitemize,2]{
        label		=	$\diamond$,
        leftmargin	=	8mm}

% Definicja wyglądu znacznika trzeciego poziomu
\setlist[myitemize,3]{
        label		=	$\diamond$,
        leftmargin	=	12mm
}

\usepackage{graphicx}   % Dołączanie rysunków
\usepackage{caption}    % Figury i przypisy
\usepackage{subcaption} % Figury i przypisy
\usepackage{footnote}   % Umożliwia tworzenie przypisów wewnątrz środowisk
\usepackage{dirtree}    % Umożliwia tworzenie struktur katalogów
\usepackage{multirow}   % Rozciąganie komórek tabeli na wiele wierszy
\usepackage{calc}       % Precyzyjne obliczenia szerokości/wysokości dowolnego fragmentu wygenerowanego przez LaTeX
\newcommand{\RR}{\mathbb{R}}    % Skrócony symbol liczb rzeczywistych
\newcommand{\NN}{\mathbb{N}}    % Skrócony symbol liczb naturalnych
\newcommand{\QQ}{\mathbb{Q}}    % Skrócony symbol liczb wymiernych
\newcommand{\ZZ}{\mathbb{Z}}    % Skrócony symbol liczb całkowitych
\newcommand{\IMP}{\rightarrow}  % Skrócony symbol logicznej implikacji
\newcommand{\IFF}{\leftrightarrow}      % Skrócony symbol  logicznej równoważności


\newtheorem{theorem}{Twierdzenie}[chapter]      % Środowisko do twierdzeń
\newtheorem{lemma}{Lemat}[chapter]              % Środowisko do lematów
\newtheorem{example}{Przykład}[chapter]         % Środowisko do przykładów
\newtheorem{corollary}{Wniosek}[chapter]        % Środowisko do wniosków
\newtheorem{definition}{Definicja}[chapter]     % Środowisko do definicji
\newenvironment{proof}{                         % Środowisko do dowodów
        \par\noindent \textbf{Dowód.}
}{
\begin{flushright}
        \vspace*{-6mm}\mbox{$\blacklozenge$}
\end{flushright}
}

\newenvironment{remark}{                        % Środowisko do uwag
        \bigskip \par\noindent \small \textbf{Uwaga.}
}{
\begin{small}
        \vspace*{4mm}
\end{small}
}

\hyphenation{wszy-stkich ko-lu-mnę każ-da od-leg-łość
        dzie-dzi-ny dzie-dzi-na rów-nych rów-ny
        pole-ga zmie-nna pa-ra-met-rów wzo-rem po-cho-dzi
        o-trzy-ma wte-dy wa-run-ko-wych lo-gicz-nie
        skreś-la-na skreś-la-ną cał-ko-wi-tych wzo-rów po-rzą-dek po-rząd-kiem
        przy-kład pod-zbio-rów po-mię-dzy re-pre-zen-to-wa-ne
        rów-no-waż-ne bi-blio-te-kach wy-pro-wa-dza ma-te-ria-łów
        prze-ka-za-nym skoń-czo-nym moż-esz na-tu-ral-na cią-gu tab-li-cy
        prze-ka-za-nej od-po-wied-nio}


\usepackage{tikz}
\usetikzlibrary{automata,positioning,fit,arrows}

\lstdefinestyle{customc}{
  belowcaptionskip=1\baselineskip,
  breaklines=true,
  frame=L,
  xleftmargin=\parindent,
  language=C,
  showstringspaces=false,
  basicstyle=\footnotesize\ttfamily,
  keywordstyle=\bfseries\color{green!40!black},
  commentstyle=\itshape\color{purple!40!black},
  identifierstyle=\color{blue},
  stringstyle=\color{orange},
}

\frontmatter
\begin{document}
        \begin{titlingpage}
                \vspace*{\fill}
                \begin{center}
                        \begin{picture}(300,510)
                                \put(11,520){\makebox(0,0)[l]{\large \textsc{Wydział Podstawowych Problemów Techniki}}}
                                \put(11,500){\makebox(0,0)[l]{\large \textsc{Politechnika Wrocławska}}}
                                \put(-10,320){\Huge \textsc{Algorytmy filtrowania w}}
                                \put(-15,290){\Huge \textsc{programowaniu ograniczeń}}
                                \put(90,250){\makebox(0,0)[l]{\large \textsc{Tomasz Kulik}}}
                                \put(200,100){\makebox(0,0)[l]{\large Praca magisterska napisana}}
                                \put(200,80){\makebox(0,0)[l]{\large pod kierunkiem}}
                                \put(200,60){\makebox(0,0)[l]{\large dr Przemysława Kobylańskiego}}
                                \put(115,-70){\includegraphics[width=0.15\textwidth]{graphics/pwr}}
                                \put(106,-80){\makebox(0,0)[bl]{\large \textsc{Wrocław 2019}}}
                        \end{picture}
                \end{center}
                \vspace*{\fill}
        \end{titlingpage}
        \cleardoublepage
        \pagenumbering{Roman}
        \pagestyle{tableOfContentStyle}
        \tableofcontents
        \cleardoublepage
        \pagestyle{custom}
        \mainmatter
        \chapter{Wstęp}
\thispagestyle{chapterBeginStyle}


\par
Praca swoim zakresem obejmuje wstęp do tematyki globalnych ograniczeń w programowaniu z ograniczeniami oraz propozycję
nowego sposobu zapisywania ograniczeń na sekwencjach zmiennych. Celem niniejszej pracy jest zaprojektowanie oraz implementacja
nowych algorytmów filtrowania w programowaniu z ograniczeniami oraz porównanie ich efektywności z innymi rozwiązaniami. 
Implementacja została wykonana w języku C++11 przy użyciu bibliotek z pakietu IBM ILOG CPLEX CP Optimizer. W pracy zostały 
przedstawione podstawowe pojęcia pozwalające na zrozumienie zagadnienia programowania z ograniczeniami oraz innych
wykorzystanych koncepcji. Pierwszy rozdział stanowi krótki wstęp do problematyki programowania z ograniczeniami.
Omówiony został w nim mechanizm poszukiwania rozwiązań dopuszczalnych oraz propagacji ograniczeń. Głównym mechanizmem,
na którym skupia się praca jest możliwość zdefiniowania ograniczenia w oparciu o skonstruowane przez użytkownika
wyrażenie regularne. Rozdział drugi zawiera opis działania algorytmu pozwalającego na wprowadzanie ograniczeń opisanych
wyrażeniem regularnym. Dzięki połączeniu zmodyfikowanej metody Thompsona kompilacji \textit{regex} do automatów NFA z algorytmem
pozwalającym na przetworzenie sekwencji zmiennych oraz filtrowanie ich dziedzin umożliwiono wprowadzenie wygodniejszej
w zdefiniowaniu metody. Nie jest to jednak główna zaleta płynąca z takiego podejścia. Algorytm filtrujący może wykorzystać
podane w wyrażeniu informacje dotyczące akceptowanych ciągów i w każdej iteracji odrzucać nieosiągalne wartości z dziedzin
zmiennych. W ostatniej części zaprezentowany został sposób użycia nowego ograniczenia w przykładowych modelach. Przedstawiono wyniki
analizy wydajności rozwiązanych modeli wykorzystujących wspomniany sposób oraz porównano je z modelami, w których
wykorzystano inne predefiniowane ograniczenia dostępne w pakiecie.

        \cleardoublepage
        \chapter{Podstawowe pojęcia}
\thispagestyle{chapterBeginStyle}
\label{rozdzial1}
Rozdział poświęcony jest między innymi krótkiemu wprowadzeniu w tematykę programowania z ograniczeniami, działania
algorytmów filtrujących oraz ich miejsca w modelowaniu. Omówiono również podstawowe zagadnienia z dziedziny języków
formalnych jakimi są języki regularne oraz praktyczne sposoby ich opisu. Połączenie powyższych zagadnień pozwala
na opis globalnych ograniczeń przy pomocy wyrażeń regularnych na sekwencjach zmiennych o określonych dziedzinach.


\section{Programowanie z ograniczeniami}
\par
Nawiązując do książki "Handbook of Constraints Programming"~\cite{HandbookOfCP}, programowanie z ograniczeniami to
paradygmat służący rozwiązywaniu kombinatorycznych problemów przeszukiwania, pojawiających się w wielu dziedzinach
techniki m. in. w sztucznej inteligencji, szeroko pojętej informatyce oraz badaniach operacyjnych.
\par
Formalnie problem spełnienia ograniczeń $P$ to taka trójka $(X, D, C)$, gdzie 
\begin{itemize}
    \item $X$ - n-elementowa krotka $(X_1, X_2, ... , X_n)$ zmiennych,
    \item $D$ - n-elementowa krotka $(D_1, D_2, ... , D_n)$ zbiorów stanowiących dziedziny odpowiadających zmiennych z
                krotki X, tj. $X_i \in D_i$,
    \item $C$ - t-elementowa krotka $(C_1, C_2, ... , C_t)$ ograniczeń narzuconych na zmienne z krotki $X$.
\end{itemize}
Ograniczenia są to relacje pomiędzy zmiennymi. Innymi słowy ograniczenie $C_i$ jest to para $(R_{Si}, S_i)$, gdzie
\begin{itemize}
    \item $S_i$ jest krotką złożoną ze zmiennych krotki $X$,
    \item $R_{Si}$ jest podzbiorem iloczynu kartezjańskiego dziedzin zmiennych
        z krotki $S_i$, tj. $R_{Si} \subseteq (D_1^{Si} \times D_2^{Si} \times ... \times D_k^{Si})$.
\end{itemize}
Rozwiązaniem problemu $P$ jest n-elementowa krotka $A = (a_1, a_2, ... , a_n)$, w której każde $a_i \in D_i$ oraz
wszystkie narzucone ograniczenia zostały spełnione - $(\forall_{(R, S) \in C}) \pi_S(A) \in R$, gdzie
$\pi_S(A)$ jest to rzut krotki $A$ zawierający zmienne ze zbioru $S$. Zbiorem $sol(P)$ nazywa się zbiór rozwiązań
dopuszczalnych dla problemu $P$ zawierającym wszystkie możliwe rozwiązania spełniające ograniczenia.
\par
W dalszej części rozważany będzie również zbiór $\Omega = D_1 \times D_2 \times ... \times D_n$.
Jest to zbiór wszystkich krotek, w których wartość każdej zmiennej przynależy do odpowiedniej dziedziny.
Innymi słowy jest to zbiór dopuszczalnych wyników ze względu na dziedziny zmiennych.
\par
\begin{example}
\textbf{Problem spełnienia ograniczeń:}
\begin{equation}
    \left\{
        \begin{aligned}
            X &= (A, B, C), \\
            D &= (\{2,3\}, \{3,4\}, \{8,16,32\})
        \end{aligned}
    \right.
\end{equation}
Powyżej zdefiniowane zostały zmienne oraz ich dziedziny. Następnym krokiem jest określenie ograniczeń na tych zmiennych:
\begin{equation}
    \left\{
        \begin{aligned}
            3A &> C \\
            3B &> C \\
            A &\neq B,
        \end{aligned}
    \right.
\end{equation}
Jedynym dopuszczalnym rozwiązaniem dla powyższego problemu jest trójka: $(A=3, B=4, C=8)$.
\end{example}

\section{Poszukiwanie rozwiązania}
\par
Podstawą działania algorytmów rozwiązujących problemy programowania z ograniczeniami są:
\begin{itemize}
    \item Przeszukiwanie
    \item Wnioskowanie
\end{itemize}
Pierwsze z nich polega kolejnym generowaniu potencjalnych wektorów wynikowych 
$A' \in \Omega$. Jeśli krotka $A'$ spełnia wszystkie narzucone ograniczenia z $C$
to staje się jednym z potencjalnie wielu rozwiązań danego problemu. Stosuje się różne techniki
oraz heurystyki poszukiwania rozwiązania.
\par
Łatwo zauważyć, że przetestowanie wszystkich
możliwych krotek wymaga $|\Omega| = |D_1| * |D_2| * ... * |D_n|$ prób. Celem zredukowania mocy zbioru $\Omega$
stosuje się metodę wnioskowania, tj. usuwania z dziedzin zmiennych tych wartości, które z pewnością nie mogą zostać
wybrane ze względu na narzucone ograniczenia. Taki proces nazywa się propagacją ograniczeń. Również w tej kwestii
rozważyć można wiele różnych technik. W ogólności im bardziej algorytm jest w stanie zawęzić dziedziny tym
mniej pracy będzie miał algorytm przeszukiwania, co najczęściej przyspiesza znalezienie wyniku.
\par
Połączenie wyżej wspomnianych technik skutkuje szybszym znalezieniem rezultatów. Iteracyjne stosowanie przeszukiwania
z nawrotami oraz propagacji ograniczeń zawęża drzewo przeszukiwania, co jest ważnym atutem w przypadku problemów
NP-trudnych, do których najczęściej stosuje się technikę programowania z ograniczeniami.

\section{Propagacja ograniczeń}
W przypadku algorytmów przeszukiwania można uciekać się do stosowania różnych heurystycznych metod. Jednak
to zawężanie zbioru dopuszczalnych rozwiązań stanowi główną zaletę algorytmu. Podstawową metodą redukcji dziedzin
jest propagacja przy użyciu tzw. Network Consistency. Ta metoda pozwala zachować lokalną spójność pomiędzy
dziedzinami zmiennych.
\par
Niech zmienne z problemu $P$ będą reprezentowane przez wierzchołki w grafie. Pierwszą z technik jest wprowadzenie
spójności wierzchołkowej (ang. Node consistency). Polega to redukcji dziedzin zmiennych tak, by spełniały ograniczenia
jednoargumentowe dla nich samych, tj. $D'_i = D_i \cap R_j$, dla każdego $R_j$ będącego ograniczeniem dla zmiennej $X_i$. 
Krawędzie pomiędzy wierzchołkami reprezentują ograniczenia binarne pomiędzy parami zmiennych. Dla ograniczenia pomiędzy
zmiennymi $X_i$ oraz $X_j$ - $R_{ij}$ należy rozpatrzeć spójność krawędziową - ang. Arc consistency.
W tym celu wykluczyć można te wartości dziedzin zmiennych, które nie należą do relacji, tj. wyznaczyć
$D'_i = D_i \cap \pi_i(R_{ij})$ oraz analogicznie $D'_j = D_j \cap \pi_j(R_{ij})$. Po zredukowaniu dziedziny
danej zmiennej może okazać się, że w grafie powstały kolejne niespójności, więc proces ten należy powtarzać iteracyjnie
do momentu osiągnięcia pełnej spójności grafu.
\par
Kolejnym możliwym etapem propagacji ograniczeń jest spójność ścieżek - ang. Path consistency. Analogicznie jak w przypadku
spójności krawędziowej można przeanalizować spójność relacji dwóch zmiennych $R_{ij}$ względem dziedziny trzeciej zmiennej
$X_m$. Polega to na wykluczaniu z $R_{ij}$ takich par dopuszczalnych wartości, które nie pozwalają zmiennej $X_m$
przypisać żadnej wartości.

\section{Globalna spójność}
\par
Osiągnięcie ogólnej krawędziowej spójności jest problemem NP-trudnym~\cite{HandbookOfCP}. Jednak w wielu
przypadkach możliwe jest wprowadzenie wyspecjalizowanych algorytmów o niższej złożoności czasowej i/lub
pamięciowej. Za sztandarowy przykład można uznać ograniczenie \textit{all\_different}, narzucające na zmienne warunek, aby
wartości tych zmiennych były parami różne - $\forall_{x'_1, x'_2 \in all\_diff(x_1,...,x_k)} x'_1 \neq x'_2$.
\begin{example}
\textbf{Problem all\_different:}
\begin{equation}
    \begin{cases}
        \begin{aligned}
            X &= (A, B, C), \\
            D &= (\{1, 2, 3\}, \{1, 2, 3\}, \{1, 2, 3\}), \\
            &\text{pod warunkiem, że:} \\
            &\text{all\_different($A$, $B$, $C$).}
        \end{aligned}
    \end{cases}
\end{equation}
Zbiorem rozwiązań są wszystkie permutacje trójki: $(1, 2, 3)$.
\par
\end{example}
W celu zrozumienia sensu wprowadzania dodatkowych algorytmów dla \textit{all\_different} należy rozpatrzeć następny przykład.
\begin{example}
\textbf{Problem zmiennych o parami różnych wartościach:}
\begin{equation}
    \begin{cases}
        \begin{aligned}
            X &= (A, B, C, D), \\
            D &= (\{1, 2, 3\}, \{1, 2, 3\}, \{1, 2, 3\}, \{1, 2, 3\}), \\
            &\text{pod warunkiem, że:} \\
            A &\neq B, A \neq C, A \neq D, \\
            B &\neq C, B \neq D, C \neq D.
        \end{aligned}
    \end{cases}
\end{equation}
Zbiór rozwiązań jest pusty, ponieważ nie jest możliwe takie wartościowanie, które dla każdej pary zmiennych przyporządkuje różne
wartości. Zauważyć można, że spójność wierzchołkowa wraz z krawędziową są zachowane, więc nie nastąpi redukcja dziedzin.
Analiza spójności ścieżek również nie wykaże problemów. Algorytm przeszukiwania przetestuje więc cały zbiór $\Omega$ zanim
stwierdzi, że problem nie ma rozwiązania. W tym przypadku dopiero sprawdzenie 4-spójności wszystkich kombinacji trzech zmiennych
$<X_i, X_j, X_m>$ względem czwartej zmiennej $X_k$ pozwoliłoby zauważyć sprzeczność, jednak w ogólności badanie n-spójności jest
problemem trudnym pod względem złożoności. W przypadku \textit{all\_different} istnieje jednak algorytm o wielomianowym czasie 
wykonywania, który skutecznie propaguje ograniczenia oraz jest w stanie stwierdzić sprzeczność na etapie samej propagacji~
\cite{Filtering}. Działa on w oparciu o rozwiązanie problemu na grafie dwudzielnym.
\end{example}

\section{Języki regularne}
Języki regularne jest to rodzina języków akceptowanych przez automaty skończone (o skończonej liczbie stanów).
Gramatykę można opisać w następujący sposób:
\begin{equation}
    \begin{aligned}
        A &\rightarrow a \text{ - symbol \textit{a} jest dowolnym terminalem} \\
        A &\rightarrow aB \text{ - symbol \textit{a} jest dowolnym terminalem, a \textit{B} nieterminalem} \\
        A &\rightarrow \epsilon \text{ - gdzie $\epsilon$ jest pustym łańcuchem znaków}
    \end{aligned}
\end{equation}
\par
Innym sposobem przedstawiania gramatyki regularnej są tzw. wyrażenia regularne (ang. regular expressions).
Przy pomocy konkatenacji, alternatywy oraz domknięcia Kleene'ego możliwe jest opisanie gramatyki języka regularnego:
\begin{itemize}
    \item $ab$ - konkatenacja oznacza, że po znaku $a$ musi pojawić się znak $b$
    \item $a|b$ - alternatywa wykluczająca, tj. w danym miejscu może pojawić się albo $a$ albo $b$
    \item $a^*$ - domknięcie Kleene'ego oznacza 0 lub więcej znaków $a$
\end{itemize}
Wyrażenia regularne wzbogacone są również o inne operatory, które w znacznym stopniu ułatwiają zapis wyrażenia, np.
\begin{itemize}
    \item $a^? \equiv (a|\epsilon)$
    \item $a^+ \equiv (aa^*)$
    \item $[a-z0-9] \equiv (a|b|c|...|z|0|1|...|9)$
    \item $[\string^a]$ - dowolny znak poza symbolem $a$
\end{itemize}
\par
Automaty skończone można podzielić na dwie klasy - automat deterministyczny ($DFA$ - ang. deterministic finite automaton)
oraz automat niedeterministyczny ($NFA$ - ang. non-deterministic finite automaton). Moc obu klas jest równa, co oznacza,
że każdą gramatykę regularną można przedstawić zarówno w postaci automatu $DFA$ jak i $NFA$~\cite{FormalLanguages1}.
Deterministyczny automat skończony jest to piątka $(Q, \Sigma, \delta, s, F)$:
\begin{itemize}
    \item $Q$ jest skończonym zbiorem stanów
    \item $\Sigma$ to alfabet wejściowy
    \item $\delta: Q \cap \Sigma \leftarrow Q$ jest funkcją przejścia pomiędzy stanami
    \item $s \in Q$ jest stanem startowym
    \item $F \subseteq Q$ jest zbiorem stanów akceptujących
\end{itemize}
Automaty skończone można przedstawić w postaci grafu, w którym wierzchołki oznaczają stany, a krawędzie
przejścia pomiędzy stanami (pod warunkiem wystąpienia odpowiedniego symbolu/symboli).
\begin{figure}
    \centering
    \resizebox{0.40\textwidth}{!}{
        \begin{tikzpicture}[>=stealth',shorten >=1pt,auto,node distance=1.0cm]
            \node[state,initial] (q_0)   {$q_0$}; 
            \node[state] (q_1) [above right=of q_0] {$q_1$}; 
            \node[state] (q_2) [below right=of q_0] {$q_2$}; 
            \node[state,accepting](q_3) [below right=of q_1] {$q_3$};
            \path[->] 
                (q_0) edge  node  {0} (q_1)
                    edge  node  [swap] {1} (q_2)
                (q_1) edge  node  {1} (q_3)
                    edge  [loop above] node {0} ()
                (q_2) edge  node [swap] {0} (q_3) 
                    edge  [loop below] node {1} ();
        \end{tikzpicture}
    }
    \caption{Przykład automatu skończonego, który jest równoważny wyrażeniu regularnemu: $(00^{*}1)|(11^{*}0)$}
\end{figure}
\par
Z formalnego punktu widzenia niedeterministyczny automat skończony $NFA$ z definicji różni się od $DFA$ funkcją
przejścia. W przypadku niedeterministycznej wersji występuje $\delta: Q \cap \Sigma \rightarrow 2^Q$, gdzie
$2^Q$ oznacza zbiór potęgowy stanów automatu. Oznacza to, że z danego stanu i symbolu wejściowego można
niedeterministycznie przejść do każdego ze stanów określonego podzbioru $Q$. Istnieje co najmniej kilka
metod radzenia sobie ze złożonością obliczeniową automatów $NFA$, co zostanie poruszone w następnym rozdziale.

\par
Symbol $\epsilon$ oznacza pusty łańcuch znaków. Oznacza to możliwość bezwarunkowego skoku do kolejnego stanu bez
pobrania symbolu z taśmy wejściowej. Przejście pomiędzy dwoma stanami $S_i \rightarrow S_j$ pod warunkiem wystąpienia
$\epsilon$ będzie w dalszej części pracy określone przez $\epsilon$-przejście, a ciąg możliwych przejść z danego stanu
$S_i$ do $S_j$ oznaczany
$S_i \rightarrow S_{k1} \rightarrow S_{k2} \rightarrow ... \rightarrow S_j \equiv S_i \rightarrow^* S_j$ będzie nazywane
$\epsilon$-ścieżką.

\section{Wyrażenia regularne jako opis globalnych ograniczeń}
\par
W literaturze można spotkać różne nazwy określające algorytmy propagacji ograniczeń jak np. algorytmy filtrujące,
globalne ograniczenia itp. Powstały również zbiory zawierające opisy działania oraz sposoby implementacji poszczególnych
globalnych ograniczeń. Jednym ze sposobów na opisanie rozwiązań dopuszczalnych dla zadanej sekwencji zmiennych jest
użycie automatu skończonego~\cite{Automata}.
Sekwencje akceptowane przez automat byłyby dopuszczalnymi rozwiązaniami dla danego globalnego ograniczenia, a sekwencja
niezaakceptowana oznaczałaby niespełnienie ograniczenia i w konsekwencji natychmiastowy nawrót algorytmu przeszukującego.
Jednocześnie na podstawie samego automatu można ograniczać dziedziny zmiennych, co pozwala na skuteczniejszą propagację
ograniczenia.
\par
Każdy automat skończony można przedstawić w postaci wyrażenia regularnego oraz każde wyrażenie regularne możne zostać
przedstawione w formie automatu skończonego~\cite{FormalLanguages2}. Zatem dla niektórych globalnych ograniczeń wygodnie
byłoby opisać ich działanie w formie wyrażenia regularnego.
\par
Jako dobry przykład może posłużyć problem rozmieszczania rozłącznych odcinków w wektorze zmiennych binarnych.
Załóżmy, że $X = (X_1, X_2, ... , X_n)$ oraz, że każda zmienna $X_i \in \{ 0, 1 \} $. Niech w sekwencji $X$ rozmieszczone
będzie $k$ odcinków o zadanych długościach $I = (I_1, I_2, ..., I_k)$ wartościowanych jedynkami, a pomiędzy odcinkami 
zmienne będą wartościowane zerem. Kolejność wystąpienia odcinków jest taka jak w wektorze $I$. 
\par
\begin{example}
\textbf{Problem rozmieszczania odcinków:}
\begin{equation}
    \begin{cases}
        \begin{aligned}
            X &= (X_1, X_2, X_3, X_4, X_5, X_6, X_7), \\
            D &= (\{0, 1\}, ... , \{0, 1\}), \\
            &\text{Niech w wektorze $X$ rozmieszczone będą odcinki $(2, 3)$.}
        \end{aligned}
    \end{cases}
\end{equation}
\par
Dopuszczalne rozwiązania powyższego problemu to:
\begin{equation}
    \begin{aligned}
        &(1, 1, 0, 1, 1, 1, 0) \\
        &(1, 1, 0, 0, 1, 1, 1) \\
        &(0, 1, 1, 0, 1, 1, 1) \\
    \end{aligned}
\end{equation}
\par
\end{example}
\par
Nawiązując do automatów skończonych, dla powyższego przykładu można skonstruować wyrażenie regularne na sekwencji $X$
opisujące ograniczenia na rozmieszczanie zadanych odcinków: $ 0^{*} 1^2 0^{+} 1^2 0^{*} $. W ogólności sekwencję
$I = (I_1, I_2, ..., I_k)$ można skonstruować na podstawie wzoru
$ 0^{*} (1^{I_1} 0^{+}) (1^{I_2} 0^{+}) ... 1^{I_k} 0^{*} $. To oznacza, że dla każdego przykładu powyższe ograniczenie
można zapisać w postaci wyrażenia regularnego, a co za tym idzie skonstruować automat $NFA$. Dalsza część pracy skupiona
jest na przetwarzaniu ciągu znaków wyrażenia regularnego na automat $NFA$ oraz zoptymalizowanej metodzie przetwarzania
sekwencji zmiennych (ich dziedzin) używając tego automatu. Przedstawiona została również metoda propagacji
ograniczeń na podstawie stworzonego automatu.

        \chapter{Analiza problemu}
\thispagestyle{chapterBeginStyle}
Rozdział zawiera opis ograniczenia filtrującego dziedziny przy pomocy automatu skończonego. Omówiono schemat tworzenia
automatu $NFA$ na podstawie wyrażenia regularnego metodą Thompsona. Sposoby opisu globalnych ograniczeń przy pomocy automatów
skończonych z licznikiem zostały wprowadzone w różnych implementacjach języka Prolog, co pozwoliło na usprawnienie procesu modelowania~
\cite{Automata}. Praca skupia się jednak na automatach skończonych, które nie zawierają liczników, co pozwala na wprowadzenie
wygodniejszego interfejsu dla użytkownika (wyrażenia regularne zamiast opis automatu skończonego) oraz większą wydajność.

\section{Założenia}
\par
Algorytm filtrujący ma za zadanie zredukować dziedziny zmiennych celem skrócenia poszukiwania dopuszczalnego rozwiązania.
Dla zadanego wyrażenia regularnego $R$ oraz sekwencji zmiennych $(X_1, X_2, ... , X_n)$ proponowany algorytm
powinien rozpropagować wśród zadanych zmiennych ograniczenie polegające na akceptacji jedynie takich krotek
$(a_1, a_2, ... , a_n)$, które są akceptowane przez automat skończony równoważny z zadanym $R$. Dla przykładu
propagacja w ciągu zmiennych binarnych $(X_1, X_2, X_3, X_4)$ wyrażenia \textit{"0* 1 1 1 0*"} powinna zawęzić dziedziny
zmiennych $X_2$ oraz $X_3$ do wartości $1$. W przypadku stwierdzenia, że dla zmiennych o danych dziedzinach nie jest
możliwe wartościowanie akceptowane przez automat, algorytm powinien zgłosić naruszenie ograniczenia.
\par
W celu zinterpretowania wyrażenia regularnego na potrzeby propagacji konieczne jest stworzenie na jego podstawie
automatu skończonego. W dalszej części rozdziału omówiony został sposób kompilacji wyrażeń regularnych oraz krótka analiza
pozytywnych oraz negatywnych cech wybranych metod. Ostatnia część zawiera znane w literaturze algorytmy
wykorzystujące automaty niedeterministyczne do testowania zadanego tekstu pod kątem akceptacji przez automat.
W kolejnym rozdziale przedstawiono modyfikacje omówionych algorytmów pozwalające na filtrowanie dziedzin zmiennych
w problemach programowania z ograniczeniami.

\section{Kompilacja wyrażenia regularnego}
\par
Istnieje co najmniej kilka sposobów na przetłumaczenie wyrażenia regularnego na automat skończony. W przypadku problemów
wyszukiwania wzorca w tekście jedną z możliwości jest kompilacja do automatu $DFA$, ta jednak wymaga asymptotycznie
wykładniczej ilości pamięci w stosunku do długości wyrażenia regularnego w najgorszym przypadku. Jest to powód, który
czyni tę metodę niepraktyczną pomimo czasu działania zależnego liniowo od długości tekstu do przeszukania.
\par
Drugą metodą jaka się nasuwa jest kompilacja do $NFA$. Przykładem może być algorytm Thompsona~\cite{Thompson}, który na podstawie wyrażenia
regularnego ,,składa'' automat niedeterministyczny przy pomocy prostych reguł zdefiniowanych dla podstawowych
operatorów języków regularnych opisanych w pierwszym rozdziale. Podręcznik "Handbook of Formal Languages vol. 2"~\cite{FormalLanguages2}
wspomina o metodzie Thompsona oraz opisuje drugą wersję pozbawioną tzw. $\epsilon$-przejść pomiędzy stanami. Dla uproszczenia analizy
w pracy tej wykorzystano jednak pierwszą z nich.
\begin{theorem}
Metoda Thompsona ma złożoność czasową i pamięciową kompilacji wyrażenia regularnego do $NFA$ ograniczoną przez $O(|r|)$.
\end{theorem}
\begin{figure}
	\centering
	\begin{center}
	\begin{subfigure}{0.45\textwidth}
		\resizebox{\linewidth}{!}{
			\begin{tikzpicture}[>=stealth',shorten >=1pt,auto,node distance=1.0cm]
				\node[state,initial] (q_0)   {};
				\node[state] (q_1) [above right=of q_0] {};
				\node[state] (q_11) [right=of q_1] {};
				\node[state] (q_2) [below right=of q_0] {}; 
				\node[state] (q_22) [right=of q_2] {};
				\node[state,accepting](q_3) [below right=of q_11] {};
				\node [draw=black!50, fit={(q_1) (q_11)}] {};
				\node [draw=black!50, fit={(q_2) (q_22)}] {};
				\path[->] 
					(q_0) edge  node  {$\epsilon$} (q_1)
						edge  node  {$\epsilon$} (q_2)
					(q_11) edge  node  {$\epsilon$} (q_3)
					(q_22) edge  node  {$\epsilon$} (q_3);
			\end{tikzpicture}
		}
		\caption{Konstrukcja alternatywy.}
	\end{subfigure}
	\end{center}
	\hspace{15pt}
	\begin{center}
	\begin{subfigure}{0.45\textwidth}
		\resizebox{\linewidth}{!}{
			\begin{tikzpicture}[>=stealth',shorten >=1pt,auto,node distance=1.0cm]
				\node[state,initial] (q_0)   {};
				\node[state] (q_1) [right=of q_0] {};
				\node[state] (q_2) [right=of q_1] {}; 
				\node[state,accepting](q_3) [right=of q_2] {};
				\node [draw=black!50, fit={(q_0) (q_1)}] {};
				\node [draw=black!50, fit={(q_2) (q_3)}] {};
				\path[->] 
					(q_1) edge  node  {$\epsilon$} (q_2);
			\end{tikzpicture}
		}
		\caption{Konstrukcja konkatenacji.}
	\end{subfigure}
	\end{center}
	\hspace{15pt}
	\begin{center}
	\begin{subfigure}{0.45\textwidth}
		\resizebox{\linewidth}{!}{
			\begin{tikzpicture}[>=stealth',shorten >=1pt,auto,node distance=1.0cm]
				\node[state,initial] (q_0)   {}; 
				\node[state] (q_1) [right=of q_0] {};
				\node[state] (q_2) [right=of q_1] {}; 
				\node[state,accepting](q_3) [right=of q_2] {};
				\node [draw=black!50, fit={(q_1) (q_2)}] {};
				\path[->] 
					(q_0) edge [bend right=35] node  {$\epsilon$} (q_3)
						edge node  {$\epsilon$} (q_1)
					(q_2) edge [bend right=35]  node  {$\epsilon$} (q_1)
						edge node  {$\epsilon$} (q_3);
			\end{tikzpicture}
		}
		\caption{Konstrukcja operatora gwiazdki - domknięcia Kleene'ego.}
	\end{subfigure}
	\end{center}
	\caption{Powyższa figura przedstawia struktury potrzebne do skonstruowania automatu $NFA$ metodą Thompsona~\cite{Thompson}.}
\end{figure}

\section{Działanie automatu}
\par
Choć sama kompilacja działa w czasie liniowym oraz wymaga wielkości pamięci liniowo zależnej od długości wyrażenia regularnego, to
kluczowym problemem jest tutaj sam algorytm przetwarzający tekst wejściowy zgodnie z automatem. Metoda przeszukiwania z nawrotami
wymaga wykładniczego czasu względem liczby stanów automatu, ponieważ każde miejsce, w którym występuje niedeterministyczne przejście
pomiędzy stanami jest przeszukiwane oddzielnie.
\par
Istnieje jednak algorytm, który ,,symuluje'' niedeterminizm występujący w automatach $NFA$ - \textit{RegularExpressionTester}~
\cite{FormalLanguages2}. Zasada jego działania polega na tym, aby utrzymywać zbiór stanów, w którym aktualnie
znajduje się automat. Innymi słowy zamiast jednego stanu w danym czasie automat może znajdować się w wielu stanach,
aktualizując każdy ze stanów w czasie wczytywania kolejnych znaków z wejścia. Wartość zwrócona przez funkcję
\textit{RegularExpressionTester} mówi o tym, czy wejściowy ciąg znaków został zaakceptowany przez automat.
\begin{theorem}
Metoda \textit{RegularExpressionTester} z podręcznika ,,Handbook of Formal Languages''~\cite{FormalLanguages2} ma złożoność czasową
zależną od długości wejścia i liczby stanów $O(n*k)$. Złożoność pamięciowa wynosi $O(k)$.
\end{theorem}
\begin{figure}
	\centering
	{\small
		\begin{pseudokod}[H]
		\SetArgSty{normalfont}
		\SetKwProg{Fn}{Function}{}{}
		\SetKwFunction{Enqueue}{Enque}
		\SetKwFunction{Dequeue}{Dequeue}
		\Fn{Closure($E$, $S$)}
		{
			R $\leftarrow$ S \\
			Q $\leftarrow$ empty queue \\
			\ForEach{$p \in S$}
			{
				\Enqueue{Q, $p$}
			}
			\While{Q is not empty}
			{
				p $\leftarrow$ \Dequeue{Q} \\
				\ForEach{state $q$ such that $(p, \epsilon, q) \in E$}
				{
					\If{$q \notin R$}
					{
						R $\leftarrow$ $R + {q}$ \\
						\Enqueue{Q, q}
					}
				}
			}
			\Return{R}
		}
		\caption{Algorytm znajdujący domknięcie danego stanu, czyli zbiór wszystkich stanów osiągalnych przez $\epsilon$-ścieżki
				od stanu wejściowego $S$.}
		\label{alg:Closure}
		\end{pseudokod}
	}
\end{figure}

\begin{figure}
	\centering
	{\small
		\begin{pseudokod}[H]
		\SetArgSty{normalfont}
		\SetKwProg{Fn}{Function}{}{}
		\Fn{Transitions($E, S, a$)}
		{
			R $\leftarrow$ empty set \\
			\ForEach{$p \in S$}
			{
				\ForEach{state $q$ such that $(p, a, q) \in E$}
				{
					R $\leftarrow$ $R + {q}$
				}
			}
			\Return{R}
		}
		\caption{Algorytm symulujący niedeterminizm automatu $NFA$ w czasie wielomianowym i pamięci ograniczonej wielomianem
				od liczby stanów.}
		\label{alg:Transitions}
		\end{pseudokod}
	}
\end{figure}

\begin{figure}
	\centering
	{\small
		\begin{pseudokod}[H]
		\SetArgSty{normalfont}
		\SetKwProg{Fn}{Function}{}{}
		\SetKwFunction{Enqueue}{Enque}
		\SetKwFunction{Dequeue}{Dequeue}
		\SetKwFunction{Closure}{Closure}
		\SetKwFunction{Transitions}{Transitions}
		\Fn{RegularExpressionTester($x$, $y$)}
		{
			$(Q, i, {t}, E)$ $\leftarrow$ $NFA$ \\
			$C$ $\leftarrow$ \Closure{$E, {i}$} \\
			\ForEach{letter $a$ from input text}
			{
				$C$ $\leftarrow$ \Closure{$E$, \Transitions{$E, C, a$}} \\
			}
			\Return{$t \in C$}
		}
		\caption{Algorytm przetwarzający tekst wejściowy zgodnie ze skonstruowanym wcześniej automatem $NFA$. Wyjściem algorytmu 
		jest wartość $TRUE$ jeśli tekst jest akceptowany przez automat lub $FALSE$ w przeciwnym przypadku.}
		\label{alg:RegularExpressionTester}
		\end{pseudokod}
	}
\end{figure}

        \chapter{Proponowany algorytm filtrowania}
\thispagestyle{chapterBeginStyle}
\par
Rozdział zawiera opis algorytmu filtrującego dziedziny zmiennych, z których składa sekwencja wejściowa. Dzięki odpowiedniej
modyfikacji metody \textit{RegularExpressionTester} algorytm jest w stanie zebrać informacje dotyczące dopuszczalnych wartości
dla każdej zmiennej. W dalszym procesie dane te są wykorzystane do narzucania możliwych wartości w dziedzinach analizowanych
zmiennych. Po zaakceptowaniu ciągu wejściowego przez automat, zaczynając od stanu końcowego, algorytm iteruje się od końca
sekwencji aż do pierwszej zmiennej kończąc na stanie początkowym. W każdym kroku tej iteracji dla każdej zmiennej zbierane
są wartości, dzięki którym możliwe było przejście do stanu końcowego. Wartości te staną się nową dziedziną dla danej zmiennej,
a dzięki zapisanym informacjom nt. tego, z którego stanu nastąpiło przejście możliwe jest odtworzenie wszystkich możliwych przejść.
\par
Bazując na literaturze oraz analizie nowego algorytmu zebrane zostały twierdzenia dotyczące teoretycznej złożoności czasowej oraz
pamięciowej proponowanej przeze mnie metody. Poddana analizie została również poprawność algorytmu.

\section{Konstrukcja automatu filtrującego dziedziny}
\par
Powyższa metoda może zostać użyta w procesie przetwarzania sekwencji zmiennych o zadanych dziedzinach. Zamiast pojedynczego
symbolu ciągu wejściowego $a_i$ można rozważyć zbiór wartości z dziedziny $D_i$, tj. zaakceptować przejście automatu do kolejnego
stanu, jeśli wymagany do przejścia symbol występuje w dziedzinie zmiennej $X_i$. Głównym celem jest jednak znalezienie możliwości
zredukowania dziedzin zmiennych. W takim przypadku ważnymi danymi, jakie algorytm musi uzyskać są dopuszczone przez automat
wartości dla każdej zmiennej z sekwencji. Aby to uzyskać konieczne jest śledzenie przejść automatu pomiędzy stanami
dla każdej zmiennej wejściowej, więc długość sekwencji staje się kolejnym mnożnikiem w złożoności pamięciowej działania
algorytmu.
\begin{figure}
	\centering
	{\small
		\begin{pseudokod}[H]
		\SetArgSty{normalfont}
		\SetKwProg{Fn}{Function}{}{}
		\SetKwFunction{Enqueue}{Enque}
		\SetKwFunction{Dequeue}{Dequeue}
		\SetKwFunction{Closure}{Closure}
		\SetKwFunction{Transitions}{Transitions}
		\Fn{evaluateAutomatonWithInputVars($Automaton$, $Vars$)}
		{
			$(Q, start, {t}, E)$ $\leftarrow$ $Automaton$ \\
			$currentStates$ $\leftarrow$ \Closure{$E, \{start\}$} \\
			$feasibleValues$ $\leftarrow$ 2D table of size [$n$][$k$] \\
			\ForEach{variable $X_i$ from input vars sequence}
			{
				\ForEach{state $c \in C$}
				{
					\ForEach{state $q$ such that $a \in D_i$ $AND$ $(p, a, q) \in E$}
					{
						$C'$ $\leftarrow$ $C' + \{q\}$ \\
						$feasibleValues[i][c]$ $\leftarrow$ $feasibleValues[i][q] + \{(p, a)\}$ \\
					}
				}
				$currentStates$ $\leftarrow$ \Closure{$E, C'$} \\
			}
			\Return{$C, feasibleValues$}
		}
		\caption{Algorytm bazujący na funkcji \textit{RegularExpressionTester}. Dodanymi przeze mnie krokiem
				 jest zapisywanie historii przejść pomiędzy stanami w czasie procesowania kolejnych zmiennych
				 z sekwencji wejściowej.}
		\label{alg:evaluateAutomatonWithInputVars}
		\end{pseudokod}
	}
\end{figure}

\section{Redukcja dziedzin}
\par
Jeśli zbiór aktualnych stanów będzie pusty zanim wszystkie zmienne zostaną przetworzone to znaczy, że aktualne dziedziny
sekwencji nie są akceptowane przez automat i algorytm propagujący zgłosi naruszenie ograniczenia. Wówczas nastąpi nawrót
(ang. backtracking) w algorytmie przeszukiwania. Taka sama sytuacja wydarzy się w przypadku, gdy przeanalizowany zostanie cały
ciąg wejściowy i zbiór stanów aktywnych nie będzie zawierał stanu końcowego (lub stanu, który jest połączony ścieżką
$\epsilon$-przejść ze stanem końcowym).
\par
Gdy ciąg zmiennych został zaakceptowany przez automat można przystąpić do redukcji dziedzin. Zaczynając od stanu końcowego $s_k$ i
ostatniej zmiennej $D_n$ z sekwencji wejściowej dzięki zapisanym informacjom o tym, z którego stanu i jakim symbolem nastąpiło
przejście można odtworzyć wszystkie możliwe wartości zaakceptowane przez automat dla $D_n$. Oznaczając taki zbiór dopuszczalnych
wartości przez $D'_n$ można z całą pewnością uznać ten zbiór jako nową dziedzinę dla zmiennej $X_n$. Przetwarzając kolejne
stany i kolejne zmienne możliwa jest redukcja dziedzin dla każdej z nich (jeśli zbiór $D'_i$ jest różny od $D_i$).
\par
W czasie przeszukiwania algorytm propagujący ograniczenie zostaje uruchomiony za każdym razem, gdy dziedziny zmiennych
mu podlegających zostają zmodyfikowane na przykład przez działanie propagacji innych ograniczeń bądź przez ustalenie
wartościowania zmiennej (które de facto jest zredukowaniem dziedziny zmiennej do jednej wartości).
\begin{figure}
	\centering
	{\small
		\begin{pseudokod}[H]
		\SetArgSty{normalfont}
		\SetKwProg{Fn}{Function}{}{}
		\SetKwFunction{resetFeasibleValuesVectors}{resetFeasibleValuesVectors}
		\SetKwFunction{evaluateAutomatonWithInputVars}{evaluateAutomatonWithInputVars}
		\SetKwFunction{findFinishStatesAfterSearch}{findFinishStatesAfterSearch}
		\SetKwFunction{reduceVarsDomains}{reduceVarsDomains}
		\SetKwFunction{violateConstraint}{violateConstraint}
		\Fn{RegexFiltering($Automaton$, $Vars$)}
		{
			$statesAfterSearching$, $feasibleValues$ $\leftarrow$ \evaluateAutomatonWithInputVars{$Automaton$, $Vars$} \\
			$finishStates$ $\leftarrow$ \findFinishStatesAfterSearch{$statesAfterSearching$, $Automaton$} \\
			\If{$finishStates$ is empty}
			{
				\violateConstraint{}
			}
			\reduceVarsDomains{$finishStates$, $feasibleValues$, $Vars$}
		}
		\caption{Zarys algorytmu propagacji ograniczenia opisanego wyrażeniem regularnym (źródło własne).}
		\label{alg:RegexFiltering}
		\end{pseudokod}
	}
\end{figure}
\par
Algorytm~\ref{alg:RegexFiltering} korzysta ze zdefiniowanej wcześniej funkcji \textit{evaluateAutomatonWithInputVars} oraz
dwóch dodatkowych \textit{findFinishStatesAfterSearch} i \textit{reduceVarsDomains}. Pierwsza z nich filtruje stany pod kątem tego,
czy w ich domknięciach znajduje się stan końcowy. Jeśli nie ma takiego stanu to oznacza, że ograniczenie nie zostało spełnione.
Druga z wymienionych funkcji jest bardziej warta uwagi, ponieważ to w niej następuje właściwa redukcja dziedzin zmiennych.
\begin{figure}
	\centering
	{\small
		\begin{pseudokod}[H]
		\SetArgSty{normalfont}
		\SetKwProg{Fn}{Function}{}{}
		\SetKwFunction{Insert}{Insert}
		\SetKwFunction{SetDomain}{SetDomain}
		\SetKwFunction{Swap}{Swap}
		\Fn{reduceVarsDomains($finishStates$, $feasibleValues$, $Vars$)}
		{
			\For{$i$ $\leftarrow$ size of variables seq. \KwTo 1}
			{
				$nextStates$ $\leftarrow$ empty set \\
				$goodValues$ $\leftarrow$ empty set \\
				\ForEach{state $\in$ finishStates}
				{
					\ForEach{$(p, a)$ $\in$ $feasibleValues$[$i$][$state$]}
					{
						\Insert{$nextStates$, $p$} \\
						\Insert{$goodValues$, $a$} \\
					}
				}
				\SetDomain{$Vars$[$i$], $goodValues$} \\
				\Swap{$finishStates$, $nextStates$} \\
			}
		}
		\caption{Algorytm redukujący dziedziny zmiennych na podstawie wyników działania automatu $NFA$.}
		\label{alg:reduceVarsDomains}
		\end{pseudokod}
	}
\end{figure}

\section{Poprawność algorytmu}
Algorytm \textit{RegexFiltering} wyklucza te wartości z dziedzin sekwencji zmiennych, które nie będą mogły być wykorzystane do 
poprawnego wartościowania zapewniającego akceptację ciągu przez automat. 
\begin{theorem}
Proponowany algorytm filtrujący \textit{RegexFiltering} zawsze się zatrzymuje.
\end{theorem}
\begin{proof}
Procedura \textit{RegexFiltering} (patrz kod \ref{alg:RegexFiltering}) wywołuje sekwencyjnie trzy funkcje, którym należy się przyjrzeć. Funkcja
\textit{evaluateAutomatonWithInputVars} zawiera trzy zagnieżdżone pętle \textit{for} o ściśle określonych warunkach zakończenia,
które w czasie działania tej funkcji są niezmienne. Kolejnym krokiem jest funkcja \textit{findFinishStatesAfterSearch}, która
jedynie wyszukuje stany końcowe w domknięciu wszystkich stanów, w jakich zatrzymał się automat. Ostatnia \textit{reduceVarsDomains}
podobnie jak pierwsza z omawianych zawiera zagnieżdżone pętle \textit{for}, których warunki zakończenia są niezależne od jej
działania i są zdefiniowane w poprawny sposób.
\end{proof}
\begin{theorem}
Proponowany algorytm filtrujący \textit{RegexFiltering} nigdy nie usunie dopuszczalnego rozwiązania.
\end{theorem}
\begin{proof}
Wynika to z samej zasady działania algorytmu. Po przeanalizowaniu całego wejścia przez automat następuje przetwarzanie
wypełnionej struktury \textit{feasibleValues} zaczynając od stanu końcowego aż do stanu początkowego.
W ten sposób algorytm będzie poruszał się jedynie po ścieżkach zaakceptowanych przez automat, nie omijając żadnych dopuszczalnych
wartości.
\end{proof}
\par
Z tego samego twierdzenia wynika, że \textit{RegexFiltering} zawsze poprawnie redukuje dziedziny - każde możliwe dopuszczalne
rozwiązanie w czasie każdej iteracji znajduje się w przefiltrowanym zbiorze dziedzin $D'_1 \times ... \times D'_n$.

\section{Złożoność obliczeniowa algorytmu}
\begin{theorem}
Złożoność czasowa algorytmu \textit{evaluateAutomatonWithInputVars} wynosi $O(n*k^2)$, a złożoność pamięciowa $O(n*k)$.
\label{theory:evaluateAutomatonWithInputVars}
\end{theorem}
\begin{proof}
Pod względem złożoności czasowej, różnicą pomiędzy \textit{evaluateAutomatonWithInputVars} a metodą
\textit{RegularExpressionTester} jest zapisywanie historii przejść pomiędzy stanami do tablicy \textit{feasibleValues}.
Ponieważ zapis ten odbywa się w czasie $O(1)$ jest on zaniedbywalny w kontekście asymptotycznej złożoności.
Jednakże, przez wprowadzenie tzw. $\epsilon$-przejść (patrz rozdział 1.) w automacie, w pesymistycznym przypadku
w czasie procesowania każdego stanu $S_i$ może okazać się, że konieczne jest przetworzenie dodatkowo $k-i$
stanów osiągalnych przez $\epsilon$-ścieżki. To oznacza, że dla $k$ stanów algorytm może przeanalizować
dla danej zmiennej $X_j$:
\begin{equation}
	\begin{aligned}
		\sum_{i=1}^{k}{(k-i)} = \frac{1}{2}(k-1)k \in O(k^2)
	\end{aligned}
\end{equation}
Z tego wynika, że przetworzenie całego ciągu wejściowego kosztuje:
\begin{equation}
	\begin{aligned}
		\sum_{j=1}^{n}{\sum_{i=1}^{k}{(k-i)}} \in O(n*k^2)
	\end{aligned}
\end{equation}
\par
Zwiększeniu względem \textit{RegularExpressionTester} podlega też złożoność pamięciowa, ponieważ algorytm ten wymaga tablicy wielkości $n*k$ wektorów
par $(p, a)$, gdzie $p$ oznacza stan z którego nastąpiło przejście, a $a$ symbol przejścia. Każdy wektor par
może w danym przebiegu algorytmu przechować maksymalnie $l = k*\alpha$, gdzie $\alpha$ jest stałą oznaczającą maksymalną
liczbę przejść, jaka może wyjść z jednego stanu. Liczba par jaka może zostać dodana do $feasibleValues[i]$ w czasie
przetwarzania zmiennej $X_i$ to również $k*\alpha$, a więc zamortyzowany koszt pamięciowy jednego wektora par dla $X_i$
wynosi $l' = \frac{k*\alpha}{k} = \alpha$, a więc $O(1)$. To oznacza, że wielkości wektorów par nie wpływają na asymptotyczną
złożoność pamięciową.
\end{proof}
\begin{theorem}
Złożoność czasowa algorytmu \textit{reduceVarsDomains} wynosi $O(n*k)$.
\label{theory:reduceVarsDomains}
\end{theorem}
\begin{proof}
Algorytm~\ref{alg:reduceVarsDomains} iteracyjnie przechodzi po wszystkich zmiennych ciągu wejściowego oraz po zbiorze
aktualnych stanów dodając do zbioru następnych stanów kolejne wpisy, których jest maksymalnie $O(k)$. Zakładając, że
implementacja zbioru pozwala na dodanie wpisu w zaniedbywalnym czasie to po złożeniu tych funkcji wynikiem jest $O(n*k)$.
\end{proof}
\par
Korzystając z powyższych twierdzeń można wyprowadzić wniosek:
\begin{theorem}
Złożoność czasowa algorytmu \textit{RegexFiltering} wynosi $O(n*k^2)$.
\label{theorem:RegexFiltering}
\end{theorem}
\begin{proof}
Złożoność algorytmu \textit{RegexFiltering} można uzyskać przez zsumowanie złożoności poszczególnych metod składowych, tj.
(patrz algorytmy \ref{theory:evaluateAutomatonWithInputVars} i \ref{theory:reduceVarsDomains}) oraz \textit{findFinishStatesAfterSearch}.
W wyniku daje to $O(n*k^2) + O(n*k) + O(k) = O(n*k^2)$.
\end{proof}

        \chapter{Implementacja}
\thispagestyle{chapterBeginStyle}
Rozdział jest opisem wykonanej implementacji do ograniczenia $RegexConstraint$. Zawarte zostały ważniejsze części kodu kompilującego
wyrażenie regularne oraz algorytmu analizującego sekwencję zmiennych pod kątem wygenerowanego automatu. Implementacja została
wykonana w języku C++11 z użyciem algorytmów zawartych w bibliotece standardowej oraz pakietu IBM ILOG CPLEX CP Optimizer dla języka
C++. Równoważną implementację można wykonać dla języka Java (ze względu na wsparcie ze strony pakietu), jednak wykracza
to poza zakres tej pracy.


\section{Struktury danych}
\par
Ważną strukturą, która reprezentuje przejście pomiędzy dwoma stanami jest \textit{Transition} (patrz kod \ref{alg:Structures}).
Przechowuje ona wartość akceptowaną przez daną relację oraz stan, do którego automat powinien przejść. Dodatkową
informacją jest pole \textit{isEmptyString}, które mówi o tym, czy dane przejście ma być tzw. $\epsilon - transition$,
czyli przejściem bezwarunkowym. Takie przejście nie pobiera kolejnego znaku z ciągu wejściowego, zamiast tego
odbywa się pobierając tzw. pusty łańcuch znaków. Intuicyjnie jeśli automat znajduje się w stanie, w którym są
$\epsilon - transition$ to może on niedeterministycznie przełączyć w stany, na które te przejścia pozwalają.
\par
W zaproponowanej implementacji strukturą reprezentującą sam stan jest wektor przejść stanowych. Kolejne indeksy
tego wektora oznaczają kolejne możliwe wyjścia z tego stanu.

\begin{figure}
\begin{lstlisting}[caption=Struktury wykorzystywane w czasie konstruowania automatu $NFA$ oraz propagacji ograniczenia.,
    style=customc, label=alg:Structures]
/*
 * Structures used by RegexConstraint
 */
typedef unsigned int StateNumber;
struct Transition
{
    int value;
    StateNumber newState;
    bool isEmptyString;
};
typedef std::vector<Transition> State;
typedef std::vector<State> StatesVector;
\end{lstlisting}
\end{figure}


\section{Główna struktura ograniczenia}
\par
Zaimplementowana klasa \textit{RegexConstraintI} jest strukturą przechowującą wektor wygenerowanych stanów,
wektor zmiennych z sekwencji wejściowej oraz wektor wartości dopuszczalnych umieszczonego tu ze względu
na optymalizację (raz przygotowana struktura, aby algorytm nie musiał alokować jej za każdym krokiem propagacji).
Wartym uwagi konstruktorem jest ten, który przyjmuje jako argumenty wyrażenie regularne oraz wektor zmiennych. W czasie jego
działania następuje konstrukcja automatu $NFA$, który będzie później wielokrotnie używany w procesie filtrowania dziedzin.
Działanie głównych funkcji/metod zawartych w tej klasie zostało opisane w rozdziale 3.

\begin{figure}
\begin{lstlisting}[caption=Główna klasa zawierająca implementację ograniczenia \textit{RegexConstraint}., style=customc]
/**
*  Class used by IBM CP solver's engine.
*/
class RegexConstraintI : public IloPropagatorI
{
public:

    /*
    * Constructor of regex constraint. It compiles regex to NFA and prepares feasibleValues vector.
    */
    RegexConstraintI(std::vector<IloIntVar> vars, std::string regex);

    /*
    * Constructor used by IBM CP engine.
    */
    RegexConstraintI(StatesVector states,
                    std::vector<StatesVector> rect,
                    std::vector<IloIntVar> vars);

    /*
    * Function used by IBM CP engine
    */
    IloPropagatorI* makeClone(IloEnv env) const override;

    /*
    * Method that is called during constraint propagation. The result of this method's work are reduced domains.
    */
    void execute() override;

private:
    void resetFeasibleValuesVectors();
    void addState(State&& state);
    int parseRegex(const std::string& str, uint index);
    std::set<StateNumber> evaluateAutomatonWithInputVars();
    std::set<StateNumber> findFinishStatesAfterSearch(std::set<StateNumber>& statesAfterSearching);
    void reduceVarsDomains(std::set<StateNumber>& finishStates);

    /*
    * A vector used to store states created during regex compilation.
    */
    StatesVector states;

    /*
    * A vector of feasible values filled during automaton evaluation. For every state and every input variable
    * the algorithm allocates a vector of states and values from wich given state was reached.
    */
    std::vector<StatesVector> feasibleValues;

    /*
    * A vector of IBM CP integer variables.
    */
    std::vector<IloIntVar> vars;
};
\end{lstlisting}
\end{figure}


\section{Parsowanie wyrażeń regularnych}
\par
Tłumaczenie wyrażeń regularnych na automaty skończone służy do wstępnego przygotowania danych dla algorytmu filtrującego.
W literaturze opisano co najmniej kilka metod, m.in. metodę Thompsona. W zaproponowanej implementacji użyta została
wspomniana metoda (jej interpretacja). Istnieją metody, które lepiej dysponują zasobami w postaci czasu działania oraz pamięci.
Pozwala to na dalszy rozwój oraz optymalizację proponowanego algorytmu.
\par
Funkcja \textit{parseRegex} (patrz kod \ref{alg:parseRegex}) służy do tłumaczenia wyrażeń regularnych na automaty skończone $NFA$. Jest to interpretacja
metody Thompsona omówionej w poprzednich rozdziałach. Ze względu na bezkontekstowość gramatyki wyrażeń regularnych spowodowanej
dodaniem zagnieżdżonych nawiasów należało wprowadzić rekurencyjne odwołania w przypadku napotkaniu symbolu ,,$($''. W celu
umożliwienia wprowadzania liczb większych niż jednocyfrowe, gramatyka zakłada, że kolejne liczby powinny być oddzielone od
siebie znakiem innym niż cyfra. Dla przykładu w przypadku, gdy użytkownik dla zmiennych $X_1, X_2, X_3$ zechce narzucić ograniczenie,
aby $X_1 = 10$, $X_2 = (22)^*$ i $X_3 = 30$, wystarczy określić wyrażenie: $10 \; 22^*30$.
\par
Wspomniana funkcja iteruje się po kolejnych symbolach z zadanego łańcucha znaków, zaczynając od danego indeksu (jako pierwsze wywołanie
metoda zaczyna analizę od początku ciągu, czyli indeksu równemu $0$). Główną częścią tej metody jest instrukcja \textit{switch},
która w przypadku napotkania danego symbolu odpowiednio go interpretuje dodając nowy stan, bądź nowe przejścia do istniejących stanów.
W przypadku wystąpienia nawiasów funkcja odpowiednio wykonuje rekurencyjnie samą siebie, gdy napotka nawias otwierający oraz powraca
do poprzedniego wywołania napotykając nawias zamykający. Za każdym takim razem funkcja dodaje do automatu stan oznaczający przejście po wszystkich
symbolach zawartych wewnątrz nawiasu oraz zwraca ostatni przetwarzany indeks, aby funkcja, która ją wywołała mogła kontynuować przetwarzanie
od tego miejsca.
\par
Gdy funkcja napotka znak gwiazdki $*$ (domknięcie Kleene'ego) to stworzy w automacie pętlę przez dodanie $\epsilon$-przejścia do początku
ostatniego bloku. Znak $?$ powoduje akceptację poprzedniego symbolu/bloku lub bezwarunkowe pominięcie tego symbolu. W przypadku,
gdy analizowanym znakiem będzie $|$ (alternatywa), algorytm zachowa się różnie w zależności od tego, czy kolejnym elementem będzie
symbol czy blok zagnieżdżony pomiędzy nawiasami.


\begin{figure}
\begin{lstlisting}[caption=Metoda kompilująca wyrażenie regularne do automatu NFA, style=customc, label=alg:parseRegex]
int RegexConstraintI::parseRegex(const std::string& str, uint index)
{
    uint prelastState = states.size()-1;
    int currentNumber = 0;
    for(uint i=index; i<str.length(); i++)
        switch(str[i])
        {
            case '(':
                prelastState = states.size();
                i = parseRegex(str, i+1);
            break;
            case ')':
                addState({Transition {-1, (StateNumber) states.size()+1, true}});
                return i;
            break;
            case '*':
                states[states.size()-1][0].newState = prelastState;
                states[prelastState].push_back(Transition {-1, (StateNumber) states.size(), true});
            break;
            case '?':
                states[prelastState].push_back(Transition {-1, (StateNumber) states.size(), true});
            break;
            case '|':
                while(i<str.length()-1 and str[i+1] == ' ')
                    i++;
                if(str[i+1] != '(')
                {
                    states[prelastState].push_back(Transition {-1, (StateNumber) states.size()+1, true});
                    addState({Transition {-1, (StateNumber) states.size()+2, true}});
                }
                else
                {
                    states[prelastState].push_back(Transition {-1, (StateNumber) states.size()+1, true});
                    addState({});
                    prelastState = states.size()-1;
                    i = parseRegex(str, i+2);
                    states[prelastState].push_back(Transition {-1, (StateNumber) states.size(), true});
                }
            break;
            default:
                if(str[i] >= '0' and str[i] <= '9')
                {
                    currentNumber = currentNumber*10 + (str[i] - '0');
                    if(i>=str.length()-1 or str[i+1] < '0' or str[i+1] > '9' )
                    {
                        prelastState = states.size();
                        addState({ Transition {currentNumber, (StateNumber) states.size()+1, false} });
                        currentNumber = 0;
                    }
                }
        }
    return str.length();
}
\end{lstlisting}
\end{figure}

        \chapter{Eksperymenty obliczeniowe}
\thispagestyle{chapterBeginStyle}
Przedstawiony w rozdziale pierwszym przykład ograniczenia opisanego wzorem wyrażenia regularnego może posłużyć np. do rozwiązywania
tzw. \textit{Nonogramów}. Jest to również dobry przykład na zaprezentowanie wydajności działania tego algorytmu filtrującego
porównując wyniki z modeli korzystających jedynie z dostępnych w pakiecie ograniczeń. W fazie testowej algorytmu dokonano również
testów poprawności algorytmu oraz sprawdzono rzeczywistą złożoność czasową względem teoretycznej.


\section{Propagacja na jednym wektorze}
\par
Podstawowymi testami wydajnościowymi dostępnymi w plikach testowych są:
\begin{itemize}
  \item Zestaw badający złożoność czasową pod kątem rosnącej długości wektora wejściowego.
  \item Zestaw badający złożoność czasową pod kątem rosnącej złożoności wyrażenia regularnego.
\end{itemize}
\par
Wyniki zostały zobrazowane na wykresach \ref{chart:vectorLength} oraz \ref{chart:regexLength}. Na pierwszym wykresie widać
kształt funkcji ograniczonej przez prostą. Można to zinterpretować jako liniowy wzrost złożoności czasowej względem długości
wektora zmiennych wejściowych. Pokrywa się to z dowodem \ref{theorem:RegexFiltering} przeprowadzonym w rozdziale 4. Drugi z wykresów przedstawia
funkcję przechylającą się łukiem ku górze, co oznacza ponad liniowy stosunek czasu propagacji do rosnącej złożoności samego wyrażenia regularnego.
Przez \textbf{złożoność} rozumie się liczbę stanów automatu $NFA$ wygenerowanego na podstawie wyrażenia oraz liczbę $\epsilon$-przejść,
jakie w nim występują. W przykładowym teście użyto generatora wyrażeń regularnych bazującego na wzorze:
\begin{equation}
\begin{aligned}
  & Gen(k) =( (0)^* | (1)^* | (2)^* | ... | (k)^* )^*
\end{aligned}
\end{equation}
\par
Funkcja $Gen(k)$ tworzy wyrażenie regularne, dla którego automat $NFA$ po przetłumaczeniu przez funkcję \textit{parseRegex}
będzie posiadał $\epsilon$-ścieżki pomiędzy każdą parą stanów. Oznacza to, że analizując kolejne wejścia automatu, algorytm
będzie musiał dla każdego z $O(k)$ stanów przeanalizować dodatkowo $k-1$ pozostałych stanów. W tym przypadku osiąga się pesymistyczną
złożoność algorytmu: $O(n*k^2)$.
\begin{figure}
\centering
\resizebox{0.65\linewidth}{!}{
\tikzset{every picture/.style={line width=0.75pt}} %set default line width to 0.75pt        
\begin{tikzpicture}[x=0.75pt,y=0.75pt,yscale=-1,xscale=1]
%uncomment if require: \path (0,509); %set diagram left start at 0, and has height of 509
%Straight Lines [id:da6884170676007542] 
\draw [color={rgb, 255:red, 245; green, 166; blue, 35 }  ,draw opacity=1 ][line width=3]    (102.85,401) -- (146,378) -- (192,343) -- (240,315) -- (282,290) -- (325,261) -- (368,242) -- (409,216) -- (451,192) -- (503,161) -- (547,137) ;
%Shape: Axis 2D [id:dp5636889612231564] 
\draw  (50,401) -- (578.5,401)(102.85,59) -- (102.85,439) (571.5,396) -- (578.5,401) -- (571.5,406) (97.85,66) -- (102.85,59) -- (107.85,66) (192.85,396) -- (192.85,406)(282.85,396) -- (282.85,406)(372.85,396) -- (372.85,406)(462.85,396) -- (462.85,406)(552.85,396) -- (552.85,406)(97.85,311) -- (107.85,311)(97.85,221) -- (107.85,221)(97.85,131) -- (107.85,131) ;
\draw   (199.85,413) node[anchor=east, scale=0.75]{20} (289.85,413) node[anchor=east, scale=0.75]{40} (379.85,413) node[anchor=east, scale=0.75]{60} (469.85,413) node[anchor=east, scale=0.75]{80} (559.85,413) node[anchor=east, scale=0.75]{100} (99.85,311) node[anchor=east, scale=0.75]{10} (99.85,221) node[anchor=east, scale=0.75]{20} (99.85,131) node[anchor=east, scale=0.75]{30} ;
% Text Node
\draw (522,443) node  [align=left] {Długość wektora};
% Text Node
\draw (67,96) node [rotate=-270] [align=left] {czas [s.]};
\end{tikzpicture}
}
\caption{Krzywa obrazująca wzrost złożoności czasowej względem długości wejściowego wektora zmiennych.
         Złożoność wyrażenia regularnego została ustalona na $Gen(700)$.}
\label{chart:vectorLength}
\end{figure}

\begin{figure}
\centering
\resizebox{0.65\linewidth}{!}{
\tikzset{every picture/.style={line width=0.75pt}} %set default line width to 0.75pt        
\begin{tikzpicture}[x=0.75pt,y=0.75pt,yscale=-1,xscale=1]
%uncomment if require: \path (0,544); %set diagram left start at 0, and has height of 544
%Straight Lines [id:da40570912889495325] 
\draw [color={rgb, 255:red, 245; green, 166; blue, 35 }  ,draw opacity=1 ][line width=3]    (102.85,401.8) -- (151.5,396) -- (192.5,387) -- (238.5,374) -- (283.5,354) -- (325.5,332) -- (370.5,305) -- (420.5,261) -- (463.5,207) -- (506.5,156) -- (560.5,42) ;
%Shape: Axis 2D [id:dp34516663752241405] 
\draw  (50,401.8) -- (578.5,401.8)(102.85,4) -- (102.85,446) (571.5,396.8) -- (578.5,401.8) -- (571.5,406.8) (97.85,11) -- (102.85,4) -- (107.85,11) (192.85,396.8) -- (192.85,406.8)(282.85,396.8) -- (282.85,406.8)(372.85,396.8) -- (372.85,406.8)(462.85,396.8) -- (462.85,406.8)(552.85,396.8) -- (552.85,406.8)(97.85,311.8) -- (107.85,311.8)(97.85,221.8) -- (107.85,221.8)(97.85,131.8) -- (107.85,131.8)(97.85,41.8) -- (107.85,41.8) ;
\draw   (199.85,413.8) node[anchor=east, scale=0.75]{40} (289.85,413.8) node[anchor=east, scale=0.75]{80} (379.85,413.8) node[anchor=east, scale=0.75]{120} (469.85,413.8) node[anchor=east, scale=0.75]{160} (559.85,413.8) node[anchor=east, scale=0.75]{200} (99.85,311.8) node[anchor=east, scale=0.75]{2} (99.85,221.8) node[anchor=east, scale=0.75]{4} (99.85,131.8) node[anchor=east, scale=0.75]{6} (99.85,41.8) node[anchor=east, scale=0.75]{8} ;
% Text Node
\draw (470,443) node  [align=left] {Złożoność wyrażenia regularnego};
% Text Node
\draw (67,96) node [rotate=-270] [align=left] {czas [s.]};
\end{tikzpicture}
}
\caption{Krzywa obrazująca wzrost złożoności czasowej względem rosnącej złożoności automatu wyrażenia regularnego.
         Długość wektora została ustalona na $300$ zmiennych.}
\label{chart:regexLength}
\end{figure}


\section{Rozwiązywanie nonogramów}
Jedną z możliwości zastosowania $RegexConstraint$ jest rozwiązywanie zagadek logicznych nazywanych \textit{Nonogramami}
(tzw. \textit{Griddlers problem}). Polega to na odtworzeniu obrazka na podstawie informacji o liczbie odcinków
oraz ich długościach w danym rzędzie/kolumnie. Obrazki mogą być wielokolorowe bądź monochromatyczne.

\begin{center}
  \begin{tabular}{|c|r|r|r|r|r|r|}
  \hline
          & \rotatebox{-60}{\bf(2)} & \rotatebox{-60}{\bf(2)} & \rotatebox{-60}{\bf(1, 4)} & 
          \rotatebox{-60}{\bf(4)} & \rotatebox{-60}{\bf(1)} & \rotatebox{-60}{\bf(4)} \\
  \hline
  {\bf(2)}    &   &   & 1 & 1 &   &   \\
  \hline
  {\bf(2, 1)} & 1 & 1 &   & 1 &   &   \\
  \hline
  {\bf(4, 1)} & 1 & 1 & 1 & 1 &   & 1 \\
  \hline
  {\bf(4)}    &   &   & 1 & 1 & 1 & 1 \\
  \hline
  {\bf(1, 1)} &   &   & 1 &   &   & 1 \\
  \hline
  {\bf(1, 1)} &   &   & 1 &   &   & 1 \\
  \hline
  \end{tabular}
  \captionof{table}{Przykład nonogramu i jego rozwiązanie.}
\end{center}
\par
Czas rozwiązywania nonogramów jest bardzo zróżnicowany. Jeśli obrazek poddaje się w czasie samej propagacji ograniczeń
zanim w ogóle algorytm wyszukiwania zacznie działać to czas działania dla obrazków nawet o boku długości 40 jest mniejszy
niż sekunda. Jeśli jednak samo wnioskowanie nie pozwala na ścisłe odnalezienie rozwiązania, tj. po etapie wstępnej propagacji
pozostaną jeszcze zmienne o nieustalonym wartościowaniu, to czas działania znacznie się wydłuża ze względu na
losowe przydzielanie wartości przez algorytm przeszukujący drzewo rozwiązań.

\begin{center}
  \begin{tabular}{|c|r|}
  \hline
  {\bf Rozmiar obrazka} & {\bf Czas rozwiązania (sek.)} \\
  \hline
  \hline
  $10 \times 10$ & $0.01$ \\
  \hline
  $15 \times 15$ & $0.02$ \\
  \hline
  $20 \times 20$ & $0.14$ \\
  \hline
  $25 \times 25$ & $2.81$ \\
  \hline
  $30 \times 30$ & $16.68$ \\
  \hline
  \end{tabular}
  \captionof{table}{Tabela przedstawiająca wyniki czasowe rozwiązywania losowo wygenerowanych nonogramów.}
\end{center}

\par
W załączniku z kodem źródłowym są podane przykłady nonogramów o rozmiarach od $6 \times 6$ aż do $50 \times 50$.
W celu porównania działania algorytmu \textit{RegexConstraint} z istniejącą implementacją predykatu \textit{automaton}
w języku \textit{SWI-Prolog} wykonano eksperymenty na wspomnianych nonogramach. Wyniki czasów działania obu algorytmów
przedstawia tabela \ref{tab:nonograms}.
\begin{center}
  \begin{tabular}{|c|r|r|}
  \hline
    {\bf Rozmiar obrazka} &
    {\bf SWI-Prolog} &
    {\bf \textit{RegexConstraint}} \\
  \hline
  \hline
  $6 \times 6$ & $0.022$ s. & ok. $0.00$ s. \\
  \hline
  $10 \times 10$ & $0.056$ s. & $0.01$ s. \\
  \hline
  $20 \times 20$ & $0.235$ s. & $0.02$ s. \\
  \hline
  $50 \times 50$ & $4.79$ s. & $0.42$ s. \\
  \hline
  \end{tabular}
  \captionof{table}{Tabela przedstawia porównanie wyników działania programu opartego o predykat \textit{automaton} napisany w SWI-Prologu
            oraz czas rozwiązywania modelu napisanego przy użyciu ograniczenia \textit{RegexConstraint}.}
  \label{tab:nonograms}
\end{center}

% TODO

        \chapter{Podsumowanie}
\thispagestyle{chapterBeginStyle}

\par
Możliwość opisu globalnych ograniczeń przy pomocy wyrażeń regularnych wyraźnie zmniejsza skomplikowanie modeli, które
dają się zapisać w takiej formie. Dobór odpowiednich algorytmów sprawia też, że tego typu modele można rozwiązać o wiele
szybciej niż przy pomocy ,,konwencjonalnych'' metod. Pomimo użycia dość prostych metod wyniki są zaskakująco dobre i to
potencjalnie może sprawić, że użycie tego typu ograniczeń stanie się bardziej rozpowszechnione. W celu dalszego zwiększenia
wydajności można poszukać lepszych, bardziej szczegółowych rozwiązań dotyczących kompilacji wyrażeń regularnych do automatów
skończonych - wiele z takich metod została już wcześniej opisana.
\par
W implementacji języka SWI-Prolog znajduje się predykat o nazwie \textit{automaton}. Jako argumenty przyjmuje opis automatu $NFA$,
wektor liczników oraz tzw. sygnatury, czyli relacje pomiędzy kolejnymi elementami ciągu wejściowego. Działanie tej metody
jest oparte o mechanizmy wnioskowania wbudowane w sam język (programowania w logice). W przedstawionej pracy została
zaprezentowana metoda napisana w ściśle imperatywnym stylu. Nie ma żadnych przeszkód, aby użytą w tym przypadku gramatykę
wyrażeń regularnych wzbogacić o użycie liczników, a sam algorytm filtrujący o ich obsługę. Pozostawia to dalszą możliwość
rozwoju tej metody.

        \pagestyle{bibliographyStyle}
        \bibliographystyle{plabbrv}
        \bibliography{bibliography}
        \thispagestyle{chapterBeginStyle}
        \addcontentsline{toc}{chapter}{Bibliografia}
        \cleardoublepage
        \appendix
        \pagestyle{appendixStyle}
        \chapter{Zawartość płyty CD}
\thispagestyle{chapterBeginStyle}
\label{plytaCD}

Płyta CD zawiera:
\begin{itemize}
    \item Pracę dyplomową w formie elektronicznej - format PDF.
    \item Kody źródłowe biblioteki zawierające implementację \textit{RegexConstraint}
    \item Przykłady użycia ograniczenia \textit{RegexConstraint}
\end{itemize}

\par
W głównym katalogu zawierającym wszystkie dodatki zawarty jest plik \textit{Makefile}. Do jego uruchomienia potrzebne
jest środowisko posiadające polecenie 'make'. W celu skompilowania plików przykładowych należy posłużyć się poleceniem:
\begin{lstlisting}[language=bash]
    $ make compile CPX_PATH=[path to IBM CPLEX] \
            SYSTEM=[name of directory with static libs in IBM CPLEX]
\end{lstlisting}
Następnie aby uruchomić pliki testowe należy użyć polecenia:
\begin{lstlisting}[language=bash]
    $ make test
\end{lstlisting}

\par
Główny katalog zawiera m. in. katalogi \textit{Src} oraz \textit{Inc} zawierające pełną implementację omawianego ograniczenia.
Katalog test zawiera pliki testowe, służące do zaprezentowania działania zaproponowanej implementacji.
W katalogu \textit{Doc} umieszczona została praca dyplomowa w wersji PDF. 

        \cleardoublepage
\end{document}

