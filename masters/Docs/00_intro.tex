\chapter{Wstęp}
\thispagestyle{chapterBeginStyle}


\par
Praca swoim zakresem obejmuje wstęp do tematyki globalnych ograniczeń w programowaniu z ograniczeniami oraz propozycję
nowego sposobu zapisywania ograniczeń na sekwencjach zmiennych. Celem niniejszej pracy jest zaprojektowanie oraz implementacja
nowych algorytmów filtrowania w programowaniu z ograniczeniami oraz porównanie ich efektywności z innymi rozwiązaniami. 
Implementacja została wykonana w języku C++11 przy użyciu bibliotek z pakietu IBM ILOG CPLEX CP Optimizer. W pracy zostały 
przedstawione podstawowe pojęcia pozwalające na zrozumienie zagadnienia programowania z ograniczeniami oraz innych
wykorzystanych koncepcji. Pierwszy rozdział stanowi krótki wstęp do problematyki programowania z ograniczeniami.
Omówiony został w nim mechanizm poszukiwania rozwiązań dopuszczalnych oraz propagacji ograniczeń. Głównym mechanizmem,
na którym skupia się praca jest możliwość zdefiniowania ograniczenia w oparciu o skonstruowane przez użytkownika
wyrażenie regularne. Rozdział drugi zawiera opis działania algorytmu pozwalającego na wprowadzanie ograniczeń opisanych
wyrażeniem regularnym. Dzięki połączeniu zmodyfikowanej metody Thompsona kompilacji \textit{regex} do automatów NFA z algorytmem
pozwalającym na przetworzenie sekwencji zmiennych oraz filtrowanie ich dziedzin umożliwiono wprowadzenie wygodniejszej
w zdefiniowaniu metody. Nie jest to jednak główna zaleta płynąca z takiego podejścia. Algorytm filtrujący może wykorzystać
podane w wyrażeniu informacje dotyczące akceptowanych ciągów i w każdej iteracji odrzucać nieosiągalne wartości z dziedzin
zmiennych. W ostatniej części zaprezentowany został sposób użycia nowego ograniczenia w przykładowych modelach. Przedstawiono wyniki
analizy wydajności rozwiązanych modeli wykorzystujących wspomniany sposób oraz porównano je z modelami, w których
wykorzystano inne predefiniowane ograniczenia dostępne w pakiecie.
