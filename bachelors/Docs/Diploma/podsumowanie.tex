\chapter{Podsumowanie}
\thispagestyle{chapterBeginStyle}

W aplikacji rozwiązującej łamigłówki geometryczne użyto połączenia dwóch nowatorskich algorytmów, które wykorzystując wykrywanie kolizji, są w stanie znajdować kombinację ruchów w celu rozwiązania zadania. Użytkownik naśladując zachowanie programu jest zdolny do rozwiązania tego typu problemów, nie znając technik ich rozwiązywania. Wyświetlanie grafiki trójwymiarowej wspieranej biblioteką OpenGL daje w rezultacie przyjazną dla użytkownika aplikację. Wieloplatformowość wspomnianej biblioteki pozwala na uruchomienie programu w każdym środowisku, które jest przez nią wspierane. Prawdopodobnie niewiele jest aplikacji pozwalających rozwiązywać tego typu łamigłówki. Z dużym prawdopodobieństwiem graniczącycm z pewnośią nie są one w tej chwili na tyle popularne, by dotarły do każdego potencjalnego użytkownika.

Jednakże algorytmy te wymagają ulepszenia. Pesymistyczny czas fazy pierwszej przekreśla szansę na rozwiązanie większości łamigłówek o wynikowym sześcianie długości powyżej 6 mniejszych kostek. Być może praktyczne wprowadzenie paradygmatu ,,constraints programming'' (w wolnym tłumaczeniu: programowania ograniczeniami) pozwoliłoby na uzyskanie znacząco lepszych rezultatów. Dodatkowo algorytm fazy drugiej należy poddać wnikliwej analizie średniej złożoności obliczeniowej oraz niezawodności w znajdowaniu rozwiązania.
