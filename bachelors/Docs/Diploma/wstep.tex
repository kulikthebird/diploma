\chapter{Wstęp}
\thispagestyle{chapterBeginStyle}
Zagadki zdają się być dziedziną przeznaczoną głównie dla osób młodych. Wiele z nich pozytywnie wpływa na rozwój człowieka, pobudza jego wyobraźnię oraz uczy logicznego myślenia. Człowiek z natury rozumny jest w stanie wymyślać i rozwiązywać tego typu problemy. Zadanie komplikuje się, jeśli takie zagadki miałaby rozwiązywać maszyna. Bowiem ilość możliwych do sprawdzenia kombinacji często przekracza możliwości współczesnych komputerów. Aby umożliwić w pewnym tylko stopniu rozwiązywanie takich  problemów przez algorytmy komputerowe, należy nałożyć odpowiednie ograniczenia na model, co wiąże się z redukcją liczby kombinacji i możliwością wprowadzenia odpowiednich optymalizacji. Problemów wymagających analizy geometrycznej jest wiele. Szczególnym typem łamigłówek są te złożone z mniejszych elementów, które odpowiednio ułożone tworzą pewne bryły. Ciekawym przykładem jest zagadka polegająca na stworzeniu sześcianu przy pomocy połączonych ze sobą mniejszych klocków. Motywacją do napisania niniejszej pracy było przełamanie kolejnej bariery, dokładna analiza problemu oraz rozwikłanie zagadki, w tym przypadku algorytmicznej.

Celem pracy jest napisanie aplikacji do rozwiązywania łamigłówek geometrycznych złożonych z połączonych ze sobą sześcianów.

W zakres prac wchodzi zaprojektowanie aplikacji niezależnej od platformy, której działanie jest skupione na rozwiązywaniu wspomnianych łamigłówek. Aplikacja ma następujące założenia:
\begin{itemize}
\item generowanie modelu,
\item rozwiązywanie postawionego problemu,
\item prezentacja wyniku.
\end{itemize}

Praca składa się z czterech rozdziałów. W każdym rozdziale znajduje się dokładny opis przedmiotu, którego dotyczy.
W rozdziale pierwszym omówiona została łamigłówka, jej konstrukcja oraz metoda jej przedstawienia w wirtualnym świece.

W rozdziale drugim przedstawiono budowę aplikacji, ukazano zależności między jej komponentami oraz  opisano działanie poszczególnych modułów. Algorytmy zostały rozpisane w pseudokodzie.

W rozdziale trzecim poruszony jest temat obsługi programu. Program dzieli swoje działanie na część rozwiązującą oraz część prezentacji.

W rozdziale czwartym została przedstawiona analiza działania programu. Podano średni czas wykonywania algorytmów rozwiązujących w zależności od wielkości i złożoności modelu.


