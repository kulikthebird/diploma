\documentclass{beamer}
\usetheme{Warsaw}
\usepackage{ragged2e}
\usecolortheme{beaver}
\title[]
{Computer modeling and solving geometry puzzles that requires collision detection}
\author[Kulik, Kobylanski]
{
    Tomasz ~Kulik and
    Przemyslaw ~Kobylanski
}
\institute[Wroclaw University of Science and Technology]
{
  Institute of Mathematics and Computer Science\\
  Wrocław University of Science and Technology\\
  POLAND
}
\date[OR 2016]
{OR, 2016}
\subject{Computer Science}


\begin{document}
    \frame
    {
        \titlepage
    }
    \frame
    {
        \frametitle{Purpose}
        The program aims to solving some types of geometry puzzles. The result of the automaton is
        3D graphic animation that shows the solution step by step.
    }

    \section{Model representation}
    \frame
    {
        \frametitle{Next module}
        \tableofcontents[currentsection]
    }

    \frame
    {
        \frametitle{Human readable file structure}
        \itemize
        {
            \item Descriptions of each type of elements.
            \item Relations between pairs of elements:
                \begin{itemize}
                    \item Relative position of element to another one,
                    \item Capabilities of movement in relation with another element.
                \end{itemize}
            \item Description of expected results.
        }
    }

    \section{Collision detector}
    \frame
    {
        \frametitle{Next module}
        \tableofcontents[currentsection]
    }

    \frame
    {
        \frametitle{Collision prediction}
        It is one of most important part of the solver. Collision detector
        predicts potential collision and inform the main solver about this possibility.
    }

    \frame
    {
        \frametitle{Collision prediction}
        Prediction of possible collision after move of model's part:
        \begin{itemize}
            \item During object rotation
            \item During object translation
        \end{itemize}
    }

    \section{Puzzle solver}
    \frame
    {
        \frametitle{Next module}
        \tableofcontents[currentsection]
    }
    \frame
    {
        \frametitle{Searching tree}
        Solving algorithm is based on searching tree.\\
        Solver is looking for result in set of all possible moves of each element in Model.
    }
    \section{Graphics representation}
    \frame
    {
        \frametitle{Next module}
        \tableofcontents[currentsection]
    }
    \frame
    {
        \frametitle{3D animation}
        Animation based on result from solver.\\
        Tool used to render graphics is OpenGL\\
    }

    \frame
    {
        \frametitle{}
        Thank you for your attention
    }


\end{document}
